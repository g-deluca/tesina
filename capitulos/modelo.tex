\chapter{El modelo de seguirdad de Android}
\label{chapter:background}

\section{Arquitectura de la plataforma}
\label{section:architecture}

El sistema operativo Android está compuesto por cinco capas de software, ordenadas en forma de pila (o
\textit{stack}), donde cada una de ellas provee un grupo de servicios a la capa inmediatamente
superior. De esta forma, se va abstrayendo progresivamente la interacción con el hardware (la base de
la pila) hasta llegar al nivel más alto en el que se ubican las aplicaciones que realizaran las tareas
requeridas por los usuarios. A continuación, analizaremos brevemente cada uno de estos niveles:

\subsubsection*{Núcleo del sistema operativo: Linux}
\label{section:architecture:kernel}
La base de la plataforma Android es el \textit{kernel} de Linux. Desde el punto de vista de la
seguridad, utilizar a un núcleo tan estudiado a lo largo de los años como pilar de la arquitectura,
ayuda a generar confianza en la misma. Además, le permite a los fabricantes de dispositivos
desarrollar controladores de hardware para un kernel ya conocido.

Una de las principales utilidades de Linux de la que Android toma ventaja es del sistema de permisos
basado en usuarios\footnote{El mismo no debe confundirse con el sistema de permisos implementado por
Android en una de las capas superiores, que es el que se formaliza y estudia en esta tesina}. Esta
característica permite que cada aplicación sea ejecutada dentro de su propia máquina virtual con un
identificador de usuario único (\texttt{UUID}) asignado a la misma. Luego, por defecto, estos
identificadores se inicializan con permisos de lectura y escritura restringidos, de manera tal que los
recursos de cada aplicación queden debidamente aislados y protegidos de potenciales \textit{malwares}.
Sin embargo, este tipo de defensa, implica la existencia de un mecanismo de validación de referencia
que arbitre las situaciones en las que una aplicación deliberadamente desee compartir recursos con
otra. Este mecanismo es implementado en las capas superiores de la arquitectura, y es el sistema
estudiado y formalizado por esta tesina.

Otras características de Linux utilizadas por Android son: la generación de subprocesos, la
administración de memora de bajo nivel y otras funcionalidades encapsuladas dentro de SELinux
(\textit{\textbf{S}ecurity \textbf{E}nhanced Linux}) como políticas de control de acceso obligatorio
(más conocidas como \textit{MAC} por sus siglas en inglés) para distinguir entre aplicaciones del
sistema y de terceros.

\subsubsection*{Capa de abstracción de hardware}
Esta capa contiene módulos de software encargados de abstraer los diferentes componentes de hardware,
como la antena de \textit{Bluetooth} o la cámara. Estas abstracciones deben ser independientes de los
\textit{drivers} que se encuentran en el núcleo del sistema, briando flexibilidad y transparencia.
Cada módulo está autocontenido en una librería, que es cargada dinámicamente cuando alguna capa
superior lo solicita.

\subsubsection*{Entorno de \textit{runtime} y bibliotecas nativas}
En esta capa encontramos el entorno de \textit{runtime} (o ART~\cite{art}, por sus siglas en inglés)
utilizado por todas las aplicaciones del sistema junto con algunas librerías nativas que la plataforma
le ofrece a los desarrolladores.

\subsubsection*{Marco de trabajo/API de la plataforma}
En esta capa se encuentran las interfaces que la plataforma le brinda a los desarrolladores de
aplicaciones para poder acceder a todas las funciones del sistema operativo. Es en esta capa donde
encontramos los servicios que implementan el mecanismo de validación de referencia mencionado
\hyperref[section:architecture:kernel]{previamente}.

\subsubsection*{Aplicaciones del sistema y de terceros}
Las aplicaciones representan el escalón final en esta pila de abstracciones, siendo los puntos de
entrada para que cualquier usuario de Android interactúe con las funciones del dispositivo. Algunas de
estas aplicaciones ya vienen preinstaladas en la plataforma (como la de mensajería SMS o la encargada
de brindar la interfaz de configuración del sistema) mientras que otras pueden ser instaladas a través
de la tienda oficial o manualmente. 


\section{Conceptos preliminares}
El sistema que arbitra el acceso a los recursos de las aplicaciones está fuertemente basado en las
interacciones entre las mismas. Por eso, en esta sección, introduciremos algunos conceptos que serán
necesarios para un mejor entendimiento del sistema en cuestión.

\subsection{Componentes de una aplicación}

\subsection{Interacción entre componentes}

\section{Mecanismo básico de protección: permisos}

\section{Cambios recientes - análisis informal}