\chapter{Formalización del sistema de permisos}
\label{chapter:formalization}

En este capítulo se describe la extensión realizada sobre una formalización previa~\cite{luna-cleiej}
del sistema de permisos de la plataforma. El objetivo de la misma es poder modelar los nuevos
comportamientos que se introdujeron con las versiones 7, 8, 9 y 10 del sistema operativo Android. Al
final del capítulo se presentan las nuevas propiedades que se han probado sobre la formalización
actualizada, poniendo el foco en aquellas estrechamente relacionadas con los cambios modelados.

\section{Lenguaje formal utilizado}
\label{section:formalization:formal-language}
La especificación del sistema se realizó dentro del framework de trabajo Coq~\cite{coq}. Coq es un
asistente de pruebas interactivo, que permite el desarrollo de programas consistentes con su
especificación. Para lograrlo, provee tres aspectos fundamentales:
\begin{enumerate}
    \item Un lenguaje de especificación que permite escribir expresiones lógicas de alto orden,
          algoritmos y teoremas;
    \item Un asistente de pruebas que permite desarrollar pruebas matemáticas verificadas;
    \item Una herramienta de extracción de programas, que permite sintetizar programas en lenguajes
          como OCaml~\cite{ocaml} o Haskell~\cite{haskell} a partir de las especificaciones formales
          escritas previamente. Los programas construidos de esta manera suelen llamarse
          \textit{programas certificados}.
\end{enumerate}

El lenguaje lógico subyacente usado por Coq es el Cálculo de Construcciones Inductivas~\cite{cic}
(también conocido como CIC, por sus siglas en inglés).

\section{Notación}
En las siguientes secciones  utilizaremos la misma notación de los trabajos
previos~\cite{luna-cleiej,betarte-2017,betarte-2016}. La misma es similar a la sintaxis de Haskell.

\subsection{Estructuras de datos y tipos generales}
Los diccionarios (más conocidos como \textit{records} por su nombre en inglés) tendrán la forma
$\{l_1: T_1, ..., l_n: T_n \}$ y notaremos el acceso a cada uno de sus elementos como $r.l_n$.
También usaremos $\{ T \}$ para notar a un conjunto que contiene elementos de tipo $T$.

Para representar tuplas utilizaremos el producto cartesiano. Por ejempo, si $A$ y $B$ son los tipos
de los elementos de una tupla de tipo $T$, lo notaremos como $ T = (A * B)$.

Finalmente, para denotar mapeos entre valores (o funciones parciales), decidimos usar la palabra
clave utilizada en Coq: \textit{mapping}. En caso de querer relacionar elementos de tipo $A$ con
elementos de tipo $B$, notaremos $mapping\ A\ B$


\subsubsection*{Tipos enumerados}
Para definir enumeraciones utilizaremos los tipos inductivos nativos de Coq. Puede parecer algo
excesivo utilizar algo tan complejo como la inducción para representar una enumeración de valores,
pero es la única forma en la que puede lograrse dicha representación sin utilizar librerías o
extensiones del lenguaje. En definitiva, un tipo enumerado es un tipo inductivo cuyos habitantes
están definidos por extensión.

Por ejemplo, el nivel de protección de los permisos está definido de la siguiente manera:
\begin{flalign*}
    Inductive\ permLevel: Set\ &:= &&\\
    &|\ dangerous &&\\
    &|\ normal &&\\
    &|\ signature &&\\
    &|\ signatureOrSys &&
\end{flalign*}

\subsection{Tipos básicos de Coq}
A continuación detallaremos algunos de los tipos básicos de Coq más utilizados en esta tesina.

\subsubsection*{Tipo \textit{option}}
\begin{flalign*}
    Inductive\ option\ T\ &:= &&\\
    &|\ Some\ :\ T \rightarrow\ option\ T&&\\
    &|\ None\ :\ option\ T &&
\end{flalign*}


El tipo \texttt{option}, análogo a la mónada \texttt{Maybe} de Haskell~\cite{maybe-haskell}, sirve
para representar la posible ausencia de valores.

\subsubsection*{Tipo \textit{list}}
\begin{flalign*}
    Inductive\ list\ T\ &:= &&\\
    &|\ \texttt{(::)}\ :\ T \rightarrow\ list\ T\ \rightarrow\ list\ T\ &&\\
    &|\ nil\ :\ list\ T &&
\end{flalign*}

El tipo utilizado para denotar una lista de elementos. Está definido de manera inductiva, siendo
\texttt{::} un operador de punto fijo.

\subsubsection*{Tipo \textit{nat}} Utilizado para denotar a los números naturales.

\subsubsection*{Tipo \textit{Prop}}

Este tipo representa uno de los dos ``universos'' donde viven las proposiciones lógicas en Coq. El
tipo $Prop$ es impredicativo y por lo tanto, una proposición de este tipo no contiene valor
computacional~\cite{proof-irrelevance} y puede ser descartada a la hora de extraer un programa.
Utilizaremos este tipo para escribir teoremas que nos permitan razonar sobre el modelo. El otro
universo que contiene proposiciones lógicas se llama $Set$ y las proposiciones de este tipo sí
tienen valor computacional y deben ser preservadas en los programas extraídos.


\section{Estado del sistema}
\label{section:formalization:state}

\subsection{Formalización de los componentes básicos}
En esta sección presentaremos las estructuras de datos utilizadas para formalizar los distintos
componentes presentados en la sección \ref{section:preliminary}. El orden en el que introduciremos
dichas abstracciones será incremental con respecto a la complejidad de las mismas y por lo tanto,
diferirá del orden en el que los componentes fueron presentados anteriormente.

\subsubsection*{Tipos atómicos}
En la tabla \ref{table:atomic_types} se enumeran las tipos de datos atómicos. Entendemos por tipo de
dato atómico a aquellos que representan entidades básicas o bien, a aquellos cuya definición queda
sujeta a cuestiones de implementación. En el asistente de pruebas Coq, los tipos atómicos fueron
definidos como parámetros de tipo $Set$.


\begin{table}[thb!]
    \centering
    \begin{tabularx}{\linewidth}{|l X|}
        \hline
        \textbf{Nombre del tipo} & \textbf{Descripción}                                                                                                  \\
        \hline
        $idApp$                  & Identificadores/nombres de las aplicaciones.                                                                          \\
        \hline
        $idPerm$                 & Identificadores/nombres de los permisos.                                                                              \\
        \hline
        $idGrp$                  & Identificadores/nombres de los grupos de permisos.                                                                    \\
        \hline
        $idCmp$                  & Identificadores/nombres de los componentes de una aplicación.                                                         \\
        \hline
        $iCmp$                   & Identificadores/nombres de las instancias en ejecución de los componentes de una aplicación.                          \\
        \hline
        $uri$                    & Identificadores de los recursos guardados en los proveedores de contenido.                                            \\
        \hline
        $res$                    & Recursos definidos por los proveedores de contenido.                                                                  \\
        \hline
        $mimeType$               & Parámetro utilizado para representar los distintos tipos de
        MIME\cite{mime-types}. Este tipo de datos es utilizado por los $intents$ y los filtros de
        $intents$ para saber cómo decodificar la información allí almacenada.                                                                            \\
        \hline
        $Category$               & Categoría a la que puede pertenecer un $intent$.                                                                      \\
        \hline
        $Cert$                   & Certificados con los que se firman las aplicaciones.                                                                  \\
        \hline
        $Extra$                  & Abstracción usada para representar el campo de información "extra" que puede enviarse a través de los $intents$.      \\
        \hline
        $Flag$                   & Banderas que pueden ser activadas por el emisor de un $intent$ para indicar cómo debe leerse/manejarse ese mensaje.   \\
        \hline
        $SACall$                 & Llamadas a la API de Android que interfieren con el modelo de seguridad (por ejemplo, para tener acceso a internet).  \\
        \hline
        $Val$                    & Tipo genérico para representar valores. Usado para representar los valores de los recursos ($res$) de una aplicación. \\
        \hline
        $vulnerableSdk$          & SDK a partir del cual las aplicaciones empiezan a considerarse como
        $legacy$ y requieren una verificación del usuario antes de ser ejecutadas por primera vez.                                                       \\
        \hline
    \end{tabularx}
    \caption{Tipos atómicos}
    \label{table:atomic_types}
\end{table}


\subsubsection*{Permisos}
Los permisos descriptos en la sección \ref{section:android:permissions} fueron formalizados como un
registro de tres campos: el nombre o identificador del permiso en cuestión, el nombre o
identificador del grupo al que pertenece el permiso (en caso de que no pertenezca a ningún, lo
representaremos mediante el valor \textit{None}) y el nivel de protección del mismo. Formalmente,

\begin{align*}
    Perm\ :=\ \{ idP: idPerm;\ maybeGrp: option\ idGrp;\ pl: permLevel \}
\end{align*}

El tipo de datos \textit{permLevel} es un tipo de datos enumerado cuyos valores posibles son:
\textit{normal}, \textit{dangerous} y \textit{signature}. Cada uno de estos valores representa uno
de los niveles de protección descriptos en la sección \ref{section:android:permissions}.

\subsubsection*{Intents}
De manera análoga a los permisos (y a la mayoría de los componentes que mencionaremos), también
utilizamos registros para formalizar a los \textit{intents}. En este caso, los registros poseen una
mayor cantidad de campos, algunos de ellos representados con estructuras complejas como un nuevo
registro. Los nueve campos que definen a un \textit{intent} son:

\begin{itemize}
    \item \textit{idI}, el nombre o identificador del intent.
    \item \textit{cmpName}, en caso de que el intent esté dirigido a un componente en particular, el
          identificador del mismo.
    \item \textit{intType}, define el caso de uso del intent. Puede ser: iniciar una actividad,
          iniciar un servicio o transmitir un evento por \textit{broadcast} a cualquier aplicación
          interesada.
    \item \textit{action}, la acción que ejecutará el intent. Los valores posibles de este campo son
          genéricos; por ejemplo, tenemos la acción de ``ver'' (representada por la constante
          \texttt{ACTION\_VIEW}) o la acción de ``enviar'' (representada por \texttt{ACTION\_SEND}).
          Para definir qué dato puede verse o enviarse y para definir qué componente será el
          encargado de hacerlo, deben utilizarse otros datos presentes en el \textit{intent}
          (\textit{data} y \textit{category}, respectivamente).
    \item \textit{data}, contiene la información necesaria para identificar los datos que son
          necesarios para ejecutar la acción en cuestión.
    \item \textit{category}, contiene información adicional sobre la acción y el componente que
          puede llevarla a cabo. Es utilizado principalmente en los intents implícitos.
    \item \textit{extra}, contiene información adicional de cualquier tipo.
    \item \textit{flags}, banderas que contienen información sobre cómo manejar el \textit{intent}.
    \item\textit{brperm}, en caso de que el componente que recibe el intent necesite un permiso para
          ejecutarlo, el mismo deberá estar listado aquí.
\end{itemize}


Formalmente, un \textit{intent} está definido con la siguiente estructura:

\begin{align*}
    Intent\  & :=\ \{                       \\
             & idI: idInt;                  \\
             & cmpName: option\ idCmp;      \\
             & intType: intentType          \\
             & action: option\ intentAction \\
             & data: Data                   \\
             & category: list\ Category     \\
             & extra: option\ Extra         \\
             & flags: option\ Flag          \\
             & brperm: option\ Perm         \\
    \}
\end{align*}

La estructura $Data$ está definida a su vez como un record que contiene un URI al recurso que
obtendrá quien reciba el intent, el tipo de dato al que se intentará acceder y, opcionalmente, el
tipo de dato en formato MIME\footnote{
    Conocido de esta manera por sus siglas en inglés (\textit{Multipurpose Internet Mail
        Extensions}). Es un estándar que se utiliza para indicar la naturaleza y el formato de un
    documento.
}.

\begin{align*}
    Data\ :=\ \{ path: option\ uri;\ type: dataType;\ mime: option\ mimeType \}
\end{align*}

La estructura $dataType$ es una enumeración que contiene los siguientes valores: $content$, $file$ y
$other$.

\subsubsection*{Filtros de intents}
Los filtros de intents están definidos como registros de tres valores. Cada uno de ellos actúa como
un filtro sobre algún campo del intent en cuestión. Formalmente:

\begin{align*}
    IntentFilter\  & :=\ \{                         \\
                   & actFilter: list\ intentAction; \\
                   & dataFilter: list\ Data         \\
                   & catFitler: list\ Category      \\
    \}
\end{align*}

Coloquialmente, al construir un elemento de tipo \texttt{IntentFilter} construiremos un filtro que
establecerá qué tipos de acciones pueden ejecutarse sobre qué datos. El ``tipo de acción'' se define
en conjunto entre la acción y la categoría. Por ejemplo, si una aplicación define un filtro de
\textit{intents} donde la categoría es \texttt{CATEGORY\_BROWSABLE} y la acción es
\texttt{ACTION\_VIEW}, entonces estará en condiciones de ser elegida para abrir un \textit{link} que
ha sido clickeado desde un navegador. En una implementación real, los filtros de intents pueden
construirse con más campos. Sin embargo, para este modelo elegimos estos tres ya que los mismos
representan las características definitorias de un intent implícito~\cite{android-intents}.

\subsubsection*{Componentes de una aplicación}
Como mencionamos previamente en la sección \ref{section:preliminary:components}, existen cuatro
tipos de componentes que pueden conformar a una aplicación de Android: actividades, servicios,
receptores de anuncios y proveedores de contenido. Los tres primeros fueron definidos análogamente.
Todos están representados por un registro de cuatro campos que contiene: un identificador, un
indicador sobre si el componente está exportado o no (es decir, si componentes de otras aplicaciones
pueden interactuar con el mismo), el permiso (en caso de existir) que protege al componente y una
lista de los filtros de \textit{intents} definidos por el mismo. A modo de ejemplo, mostraremos la
definición formal de una actividad:

\begin{align*}
    Activity\  & :=\ \{                        \\
               & id: idCmp;                    \\
               & exp: bool                     \\
               & perm: option\ Perm            \\
               & intFilter: list\ IntentFilter \\
    \}
\end{align*}

Para el caso de los proveedores de contenidos, el registro se extiende con los siguientes valores:

\begin{itemize}
    \item \textit{readPerm} y \textit{writePerm}, permitiendo una mayor granularidad en los permisos
          necesarios para leer o escribir del mismo. En caso de que estos valores existan, tienen
          prioridad por encima de \textit{perm}.
    \item \textit{grantU} (nombrado así para acortar el nombre \textit{grantUriPermissions}), un
          campo de valor \textit{booleano} encargado de indicar si el componente tiene la capacidad
          de delegar permisos de escritura/lectura sobre sus recursos.
    \item \textit{map\_res}, un mapa que asocia URIs con recursos.
    \item \textit{uri}, una lista con todos los URIs pertenecientes al proveedor de contenido. El
          valor de este campo coincide con el listado de las claves del mapa definido anteriormente.
          Sin embargo, decidimos agregarlo al componente para simplificar algunas demostraciones.
\end{itemize}


\subsubsection*{Manifiesto}
Como mencionamos previamente en \ref{section:preliminary:manifest}, el manifiesto de una aplicación
contiene información estática sobre la misma. Nuestra representación de un manifiesto está acotada a
los datos relevantes para las propiedades demostradas. Fue definido a través de un registro de seis
valores, conformado por:

\begin{itemize}
    \item \textit{cmp}, aquí se listan los componentes que conforman la aplicación.
    \item \textit{targetSdk}, la versión de Android para la cual fue pensada/diseñada la aplicación.
    \item \textit{minSdk}, la versión mínima de Android necesaria para poder ejecutar la aplicación.
          Aquellos dispositivos que cumplan con este valor pero no con el anterior, podrán ejecutar
          la aplicación con capacidades reducidas.
    \item \textit{use}, aquí se declaran los permisos que la aplicación necesitará para funcionar
          correctamente. Los permisos que no estén aquí declarados no podrán ser otorgados a la
          aplicación bajo ninguna circunstancia.
    \item \textit{usrP}, aquí se listan los permisos declarados por la aplicación.
    \item \textit{appE}, aquí se declara, si corresponde, un permiso de seguridad que se puede
          utilizar para limitar el acceso a funciones o componentes específicos de esta aplicación.
\end{itemize}

Formalmente,

\begin{align*}
    Manifest\  & :=\ \{                 \\
               & cmp: list\ Cmp;        \\
               & minSdk: option\ nat    \\
               & targetSdk: option\ nat \\
               & use: list\ idPerm      \\
               & usrP: list\ Perm       \\
               & appE: option\ Perm     \\
    \}
\end{align*}

\subsubsection*{Aplicaciones}
A diferencia de los componentes que definimos hasta ahora, las aplicaciones instaladas por el
usuario no poseen una estructura delimitada por un registro. La representación de las mismas es más
abstracta: consta de un identificador (de tipo $idApp$) y de distintos \textit{mapeos}, guardados en
el estado del sistema, hacia las distintas partes que la conforman. Por ejemplo, el estado del
sistema mantendrá una relación entre los identificadores de las aplicaciones ($idApp$) y su
manifiesto (de tipo $Manifest$). A continuación, cuando definamos el estado, se explicitarán todos
los datos asociados a las aplicaciones.

Distinto es lo que sucede con las aplicaciones pre-instaladas en el sistema. Las mismas sí cuentan
con una estructura definida por un registro, con la siguiente información: un identificador, el
certificado con el que fue firmada, el manifiesto y un listado de los recursos y permisos que la
aplicación define. Como estas aplicaciones no pueden ser desinstaladas ni modificadas, esta
información se almacenará en forma de lista en la parte estática del estado del sistema.

\subsection{Definición de estado}
Nuestra formalización del sistema de permisos de Android puede pensarse como una máquina de estado
abstracta. En este modelo, los estados del sistema están conformados por dos componentes: uno que
almacena la información dinámica del sistema, como las aplicaciones instaladas y los permisos
otorgados a las mismas; y otro que contiene información estática como el manifiesto de cada
aplicación. Formalmente:

\begin{align*}
    System\ :=\ \{ state: State;\ environment: Environment \}
\end{align*}


El componente $State$ es el que contiene la información dinámica y está conformado por los
siguientes elementos:

\begin{itemize}
    \item Una lista con los identificadores de las aplicaciones instaladas.
    \item Una lista con información de las aplicaciones \textit{legacy} que ya han sido verificadas
          por el usuario. Es un subconjunto de las aplicaciones instaladas.
    \item Un mapeo entre las aplicaciones instaladas y los permisos otorgados a cada una de ellas.
    \item Un mapeo entre las aplicaciones instaladas y los grupos de permisos para los cuales el
          usuario ha permitido el otorgamiento automático de permisos individuales.
    \item Un mapeo entre los componentes que define una aplicación y las instancias en ejecución de
          ellos.
    \item Un mapeo entre los recursos de un proveedor de contenidos para una aplicación y los
          permisos permantentes otorgados sobre el mismo.
    \item Un mapeo entre los recursos de un proveedor de contenidos para una aplicación y los
          permisos temporales otorgados sobre el mismo.
    \item Un mapeo indicando el valor de cada uno de los recursos de una aplicación.
    \item Una lista de los intents que han sido enviados, junto con su emisor.
\end{itemize}

Por otro lado, el componente $Environment$ contiene la siguiente información estática:

\begin{itemize}
    \item Un mapeo que asocia a cada aplicación con su archivo de manifiesto.
    \item Un mapeo que asocia a cada aplicación con el certificado que se utilizó para firmarla.
    \item Un mapeo que asocia a las aplicaciones con los permisos definidos por ellas.
    \item Una lista de las aplicaciones pre-instaladas del sistema.
\end{itemize}

A continuación daremos la definición formal de ambos componentes. El orden en el que se definen los
campos de cada componente es el mismo que el de las numeraciones previas.
\begin{flalign*}
    State\ &:=\ \{ &&\\
    &apps:\ list\ idApp; \\
    &alreadyVerified:\ list\ idApp; \\
    &grantedPermGroups:\ mapping\ idApp\ (list\ idGrp); \\
    &perms:\ mapping\ idApp\ (list\ Perm); \\
    &running:\ mapping\ iCmp\ Cmp; \\
    &delPPerms:\ mapping\ (idApp\ *\ CProvider\ *\ uri)\ PType; \\
    &delTPerms:\ mapping\ (iCmp\ *\ CProvider\ *\ uri)\ PType; \\
    &resCont:\ mapping\ (idApp\ *\ res)\ Val; \\
    &sentIntents:\ list\ (iCmp*Intent) \\
    \}
\end{flalign*}

\begin{flalign*}
    Environment\ &:=\ \{ &&\\
    &manifest:\ mapping\ idApp\ Manifest; \\
    &cert:\ mapping\ idApp\ Cert; \\
    &defPerms:\ mapping\ idApp\ (list\ Perm); \\
    &systemImage:\ list\ SysImgApp; \\
    \}
\end{flalign*}

De aquí en adelante, al hablar del estado del sistema, nos estaremos refiriendo al componente
$System$. En caso de que querramos referirnos a alguno de sus sub-componentes seremos explícitos con
su nombre en inglés.

\subsubsection{Estados válidos del sistema}
No todos los elementos que habitan el conjunto definido anteriormente son relevantes al sistema que
queremos estudiar. Por ejemplo, no queremos trabajar sobre un estado en el que una aplicación
pre-instalada del sistema y una aplicación instalada por el usuario puedan tener el mismo
identificador.

Inicialmente, a la hora de definir los componentes, fue necesario pensar qué condiciones debían
cumplir estos estados para representar estados de Android que tengan sentido. En consecuencia,
definimos una noción de \textbf{estado válido} para restringir el universo de estados a aquellos que
cumplen ciertas condiciones que nos garantizarán que nuestros estados del modelo tienen sentido al
compararlos con estados reales del sistema. Vale aclarar, que nuestra definición de estado válido no
es completa y que, de alguna manera, está focalizada en las propiedades que probaremos luego.

Se definió formalmente un predicado $valid\_state$ que se satisface cuandos se cumplen las
siguientes condiciones:

\begin{itemize}
    \item Todos los componentes, ya sea que pertenezcan a una aplicación instalada por el usuario o
          a una aplicación pre-instalada tienen identificadores diferentes.
    \item Ningún componente pertenece a más de una aplicación.
    \item Ningún componente en ejecución es una instancia de un proveedor de contenido (los mismos
          no se \textit{ejecutan}).
    \item Todo permiso temporalmente otorgado ha sido otorgado a un componente en ejecución y es
          sobre un recurso de un proveedor de contenido existente.
    \item Todo componente en ejecución pertenece a una aplicación instalada en el sistema.
    \item Toda aplicación que establece un valor a un recurso está instalada en el sistema.
    \item El dominio de las funciones parciales que definen el $manifest$, $cert$ y $defPerms$ es el
          conjunto de todas las aplicaciones instaladas por el usuario.
    \item El dominio de las funciones parciales que definen $grantedPermGroups$ y $perm$ es el
          conjunto de todas las aplicaciones del sistema, tanto instaladas por el usuario como
          pre-instaladas.
    \item Todas las aplicaciones del sistema tienen identificadores diferentes.
    \item Todos los permisos definidos por las aplicaciones tienen identificadores diferentes.
    \item Todos los permisos otorgados existen en el sistema.
    \item Todos los intents que han sido enviados tienen identificadores diferentes.
\end{itemize}

\section{Acciones}
Modelamos las operaciones de Android que nos interesan estudiar como un conjunto de acciones
(definidas a través del tipo \texttt{Action}), donde cada una de ellas determina la manera en la que
nuestro sistema puede transicionar.  La tablas~\ref{table:actions:first} y
\ref{table:actions:second} resumen todas las acciones disponibles.

\begin{table}
    \label{table:actions}
    \vspace{0.2cm}
    \begin{tabularx}{\linewidth}{|l X|}
        \hline
        \textbf{Acción}   & \textbf{Descripción}                                                                                                                                               \\
        \hline
        $\mathtt{install}~app~m~c~res$     & Instala la aplicación con identificador $app$, cuyo manifiesto es $m$, su certificado es $c$ y la lista de recursos es $res$.                                                                                        \\
        \hline
        $\mathtt{uninstall}~app$           & Desinstala la aplicación con identificador $app$.                                                                                                                                                                    \\
        \hline
        $\mathtt{read}~ic~cp~u$            & El componente en ejecución $ic$ lee el recurso correspondiente al identificador URI $u$ del proveedor de contenido $cp$.                                                                                             \\
        \hline
        $\mathtt{write}~ic~cp~u~val$       & El componente en ejecución $ic$ escribe el valor $val$ en el recurso correspondiente al identificador $u$ del proveedor de contenido $cp$.                                                                           \\
        \hline
        $\mathtt{startActivity}~i~ic$      & El componente en ejecución $ic$ solicita comenzar la actividad especificada por el intent $i$.                                                                                                                       \\
        \hline
        $\mathtt{startActivityRes}~i~n~ic$ & El componente en ejecución $ic$ solicita comenzar la actividad especificada por el intent $i$ y espera como respuesta un token $n$.                                                                                  \\
        \hline
        $\mathtt{startService}~i~ic$       & El componente en ejecución $ic$ solicita comenzar el servicio especificado por el intent $i$.                                                                                                                        \\
        \hline
        $\mathtt{sendBroadcast}~i~ic~p$    & El componente en ejecución $ic$ envía el intent $i$ en modo \textit{broadcast}, especificando que solo los componentes que posean el permiso $p$ pueden recibirlo.                                                   \\
        \hline
        $\mathtt{sendOrdBroadcast}~i~ic~p$ & El componente en ejecución $ic$ envía el intent $i$ en modo \textit{broadcast} ordenado,  especificando que solo los componentes que posean el permiso $p$ pueden recibirlo.                                         \\
        \hline
        $\mathtt{sendSBroadcast}~i~ic$     & El componente en ejecución $ic$ envía el intent $i$ en modo \textit{sticky broadcast}.                                                                                                                               \\
        \hline
        $\mathtt{resolveIntent}~i~app$     & La aplicación $app$ vuelve al intent $i$ explícito.                                                                                                                                                                  \\
        \hline
        $\mathtt{stop}~ic$                 & El componente en ejecución $ic$ termina su ejecución.                                                                                                                                                                \\
        \hline
        $\mathtt{grantP}~ic~cp~app~u~op$   & El componente en ejecución $ic$ delega permisos permantentes a la aplicación $app$. Esta delegación autoriza a $app$ a realizar la operación $op$ en el recurso asignado al URI $u$ del proveedor de contenido $cp$. \\
        \hline
        $\mathtt{revokeDel}~ic~cp~u~op$    & El componente en ejecución $ic$ revoca los permisos otorgados al recurso $u$ del proveedor de contenidos $cp$ para realizar la operación $op$.                                                                       \\
        \hline
        $\mathtt{call}~ic~sac$             & El componente en ejecución $ic$ realiza el llamado a una función del sistema denominada $sac$.                                                                                                                       \\
        \hline
        $\mathtt{grant}~p~app$             & Otorga el permiso $p$ a la aplicación $app$ con la confirmación del usuario.                                                                                                                                         \\
        \hline
        $\mathtt{grantAuto}~p~app$         & Otorga automáticamente el permiso $p$ a la aplicación $app$ (sin requerir confirmación del usuario).                                                                                                                 \\
        \hline
        $\mathtt{revoke}~p~app$            & Revoca un permiso no agrupado $p$ de la aplicación $app$.                                                                                                                                                            \\
        \hline
        $\mathtt{revokePermGroup}~g~app$   & Revoca todos los permisos pertenecientes al grupo $g$ de la aplicación $app$.                                                                                                                                        \\
        \hline
        $\mathtt{hasPermission}~p~app$     & Chequea si la aplicación $app$ posee el permiso $p$.                                                                                                                                                                 \\
        \hline
        $\mathtt{receiveIntent}~i~ic~app$  & La aplicación $app$ recibe el intent $i$, enviado por el componente en ejecución $ic$.                                                                                                                               \\
        \hline
        $\mathtt{verifyOldApp}~app$        & El usuario verifica los permisos que han sido otorgados a la aplicación $app$. Solo se utiliza para aquellas aplicaciones que fueron instaladas previamente a la versión 6 de Android.                               \\
        \hline
    \end{tabularx}
    \caption{Acciones del sistema}
\end{table}

La semántica de las mismas está dada en términos de pre-condición y post-condición. Para ello,
definimos los predicados $Pre$ y $Post$ de manera tal que para que una acción $a$ pueda transicionar
el sistema desde un estado $s$ hacia otro estado $s'$, deberán cumplirse $Pre\ s\ a$ y $Post\ s\ s'\
    a$. Notaremos la transición de un estado a otro de la siguiente manera: $s\ \step{a}\ s'$.

A continuación y a modo de ejemplo, describiremos informalmente las acciones que han sido
introducidas o modificadas con las últimas actualizaciones del sistema.

\subsubsection{Semántica de \texttt{grant}}

Esta operación es la encargada de otorgar un permiso $p$ a una aplicación $a$. La misma ya estaba
presente en la formalización de la que se partió en esta tesina. Sin embargo, su semántica ha sido
modificada a raíz de las actualizaciones de la plataforma mencionadas en la sección
\ref{subsection:recent-changes:grouped-permissions}. En particular, ahora el sistema podrá
transicionar con esta operación solo si  el permiso $p$ no pertenece a un grupo o, en caso de que
pertenezca, es el primero del grupo en ser otorgado a la aplicación. El resto de las precondiciones
necesarias para transicionar se mantuvieron: el permiso $p$ debe existir (es decir, debe estar
definido o bien por el sistema o por alguna aplicación), debe estar declarado como usado en el
manifiesto de la aplicación $a$, debe ser un permiso peligroso y no debe haber sido otorgado a la
aplicación previamente.

Si la precondición se cumple, el sistema transicionará hacia un estado en dónde el permiso se agrega
a los permisos otorgados a la aplicación, es decir, se agrega $p$ al conjunto $perms\ a$; y en caso
de corresponder, ocurre lo análogo con el grupo de $p$ y $grantedPermGroups\ a$. El resto de los
componentes del sistema no se verán modificados.

\subsubsection{Semántica de \texttt{grantAuto}}

Para modelar en su totalidad los cambios mencionados en la sección
\ref{subsection:recent-changes:grouped-permissions}, además de los cambios en la semántica a la
operación \texttt{grant}, se introdujo una nueva acción \texttt{grantAuto}. Con esta nueva operación
se busca representar al otorgamiento automático de un permiso por parte del sistema operativo. Su
semántica difiere de la anterior en que el sistema solo podrá transicionar con \texttt{grantAuto} si
el permiso que se intenta otorgar pertenece a un grupo que el usuario ya ha autorizado. En caso de
que esto se cumpla, el modelo mutará hacia un estado en donde la aplicación en cuestión obtuvo el
permiso solicitado.

Decidimos representar el otorgamiento con consentimiento del usuario y el otorgamiento automático de
permisos con dos acciones distintas para obtener una mayor granularidad en las trazas de ejecución.
Esta decisión, en consecuencia, facilita la demostración de las propiedades que involucran
otorgamiento de permisos.

\subsubsection{Semántica de \texttt{revoke} y \texttt{revokePermGroup}}

De manera similar a lo ocurrido con \texttt{grant} y \texttt{grantAuto}, la semántica de las
operaciones  \texttt{revoke} y \texttt{revokeGroup} también se modificaron con los cambios de las
últimas versiones. Decidimos modelar estas operaciones de manera tal que se mantenga una relación
con la experiencia de usuario al revocar permsisos. De esta manera, la operación \texttt{revoke} es
la encargada de revocar permisos individuales \textbf{no} agrupados mientras que
\texttt{revokePermGroup} quita todos los permisos pertenecientes al grupo deseado. En otras
palabras, los permisos pertenecientes a algún grupo no pueden ser revocados de manera individual, el
grupo entero debe ser invalidado.

Si el sistema transiciona con \texttt{revoke}, entonces dado un permiso $p$ y una aplicación $a$,
obtendremos un estado en donde la aplicación $a$ ya no tendrá acceso a los recursos protegidos por
$p$. Análogamente, dado un permiso $g$ y una aplicación $a'$, al transicionar con
\texttt{revokePermGroup}, la aplicación $a$ ya no tendrá ningún permiso perteneciente al grupo $g$.
Además, el sistema ya no estará autorizado a otorgar a la aplicación $a$ permisos del grupo $g$ de
manera automática.


\subsubsection{Semántica de \texttt{verifyOldApp}}

Esta operación se agregó al modelo para razonar sobre el nuevo el comportamiento mencionado en la
sección~\ref{subsection:recent-changes:legacy-apps}. Dada una aplicación $a$, para poder
transicionar el sistema utilizando la operación $verifyOldApp\ a$, deben cumplirse las siguientes
condiciones: $a$ debe ser una aplicación instalada en el sistema, aún no debe haber sido ejecutada y
debe estar orientada a una versión previa a la sexta versión de Android.

En la implementación real de la plataforma, al momento de verificar una aplicación vieja se muestra
al usuario un menú con los permisos otorgados a la misma en el momento en que fue instalada, junto
con la posibilidad de elegir cuales de ellos se desea mantener y cuales revocar. Nuestra operación
$verifyOldApp\ a$ simplemente transiciona hacia un estado en donde a la aplicación $a$ se le han
revocado todos sus permisos y se ha marcado como verificada. Para modelar la acción en la que el
usuario selecciona los permisos que desea mantener mediante la interfaz ofrecida por Android,
deberemos dar una sucesión de acciones,  donde primero se verifica la aplicación y luego se otorgan
los permisos que se eligió mantener.

\subsubsection{Semántica de \texttt{receiveIntent}}

A raíz del cambio mencionado previamente y en la
sección~\ref{subsection:recent-changes:legacy-apps}, agregamos una nueva condición que deberá
cumplirse para que una aplicación pueda recibir un \textit{intent}: para que una aplicación $a$
pueda recibir el intent $i$, entonces la misma no debe ser una aplicación \textit{legacy} o, en caso
de serlo, debe haber sido previamente verificada por el usuario. De esta manera, las aplicaciones
\textit{legacy} que queden en el sistema podrán ser visibles a la hora de resolver un
\textit{intent} (es decir, un usuario podrá observarla entre las aplicaciones disponibles para
realizar determinada acción); pero en caso de elegirla, el usuario primero deberá verificar los
permisos antes de que la aplicación pueda ejecutar la acción.


\section{Ejecuciones}
Cuando el sistema intenta ejecutar una acción $a$ en un estado válido $s$, hay dos posibles
resultados. Si la precondición de la acción se cumple, el sistema transicionará hacia otro estado
$s'$ donde la postcondición de $a$ también se satisface. Sin embargo, si la precondición no se
cumple, el sistema permanecerá en el mismo estado en el que se encontraba al intentar la ejecución
de $a$ y responderá con un mensaje de error determinado por la relación $ErrorMsg$, definida a
continuación. Dados el estado $s$ y la aplicación $a$ mencionados previamente, y un código de error
$ec$, la relación $ErrorMsg s a ec$ se satisface si y sólo si el código $ec$ es una respuesta
aceptable cuando el sistema falla al ejecutar $a$ en el estado $s$.

Formalmente, las posibles respuestas de sistema se definen a través de la siguiente semántica
operacional:
\begin{displaymath}
    \begin{array}{c}
        \inference[]{$$valid\_state(s)$$ \hspace{.2cm} $$Pre(s, a)$$ \hspace{.2cm} $$Post(s, a, s')$$}{$$s\step{a/ok}s'$$}
        \hspace{0.5cm}
        \inference[]{$$valid\_state(s)$$ \hspace{.2cm} $$ErrorMsg(s, a, ec)$$}{$$s\step{a/error(ec)}s$$}
    \end{array}
\end{displaymath}

El siguiente teorema garantiza que toda ejecución mantiene la validez del estado del sistema. Su
demostración se realizó utilizando el asistente de pruebas \texttt{Coq} y puede encontrarse en el
archivo \texttt{ValidityInvariance.v}. Consiste, fundamentalmente, en un análisis por casos en la
acción a ejecutar ya que previamente se han demostrado por separado que cada acción preserva la
validez del estado.

\begin{theorem} [Las ejecuciones preservan la validez del estado]
    \label{lemma:valid-state-correct}
    \mbox{}\\
    $\begin{array}{l} \forall\ (s\ s':\AndroidState)(a:\Action) (r:Response), s\step{a/r}s'
            \rightarrow valid\_state(s')\end{array}$
\end{theorem}

Demostrar este tipo de invariantes facilita el razonamiento cuando se estudian otros comportamientos
más específicos del sistema. En particular, todas las propiedades que mencionamos a continuación
fueron probadas sobre estados válidos. De esta manera, infinidad de estados en los que la propiedad
no hubiese sido verdadera fueron automáticamente descartados, dado que no se trataban de escenarios
válidos.

\section{Propiedades del modelo}
\label{section:formalization:properties}

En esta sección presentaremos y discutiremos las propiedades que establecimos y demostramos sobre
nuestra formalización de Android. Todas las propiedades han sido demostradas utilizando el asistente
de pruebas \texttt{Coq}. Nos enfocamos en propiedades de \textit{safety}\footnote{Utilizamos el
    término en inglés para diferenciarlo de \textit{security}, dado que en la traducción al español
    mantener esa diferencia es más complejo.} aunque también formalizamos algunos comportamientos
potencialmente peligrosos que no han sido considerados en la especificación informal de la
plataforma.

En la tabla \ref{table:auxiliary_functions} introducimos algunas funciones y predicados auxiliares
que nos ayudarán a definir los teoremas presentados.

\begin{table}[thb!]
    \centering
    \begin{tabularx}{\linewidth}{|l X|}
        \hline
        \textbf{Función/Predicado}   & \textbf{Descripción}                                                                                                                                               \\
        \hline
        $getAppFromCmp(c,s)$         & Dado un componente $c$ y un estado $s$, devuelve la aplicación a la cual pertenece dicho componente.                                                               \\
        \hline
        $getAppRequestedPerms(m)$    & Dado un manifiesto $m$ de una aplicación, devuelve los permisos listados como usados.                                                                              \\
        \hline
        $getGrantedPermsApp(app,s)$  & Devuelve los permisos con los que cuenta la aplicación $app$ en el estado $s$.                                                                                     \\
        \hline
        $getAuthorizedGroups(app,s)$ & Dada una aplicación $app$ y un estado $s$, devuelve los grupos de permisos que se encuentran autorizados para otorgar permisos automáticamente a dicha aplicación. \\
        \hline
        $getManifestForApp(app,s)$   & Devuelve el manifiesto de la aplicación $app$. El estado es necesario
        como argumento porque el manifiesto se encuentra guardado en la parte estática del mismo.                                                                                                         \\
        \hline
        $getPermissionId(p)$         & Devuelve el identificador del permiso $p$.                                                                                                                         \\
        \hline
        $getPermissionLevel(p)$      & Devuelve el nivel de protección del permiso $p$.                                                                                                                   \\
        \hline
        $getPermissionGroup(p)$      & Devuelve $Some~g$ si el permiso $p$ pertenece al grupo $g$, o $None$ en caso contrario.                                                                            \\
        \hline
        $getRunningComponents(s)$    & Devuelve un conjunto de pares conformados por el ID de una instancia en ejecución con el componente asociado a la misma.                                           \\
        \hline
        $oldAppNotVerified(app,s)$   & Válido si y solo si la aplicación $app$ es considerada $legacy$ y el usuario aún no la ha verificado en el estado $s$.                                               \\
        \hline
    \end{tabularx}
    \caption{Funciones auxiliares y predicados}
    \label{table:auxiliary_functions}
\end{table}


A raíz del cambio mencionado en la sección \ref{subsection:recent-changes:grouped-permissions}, la
primer propiedad que formulamos establece una condición necesaria para que nuestra formalización del
sistema sea consecuente con la documentación de la plataforma: solo los permisos que pertenecen a un
grupo ya autorizado por el usuario pueden ser otorgados de manera automática por el sistema.

\begin{prop} \label{section:formalization:property1} \mbox{} \\ \\
    $\forall (s,s': \AndroidState) (p:\Perm) (g:\PermGrp) (app:\AppId),$ \\
    $getPermissionLevel(p) = dangerous \land getPermissionGroup(p) = Some\ g\ \land g \notin
        getAuthorizedGroups(app,s) \rightarrow\ \neg
        \Mathexecrel{s}{\texttt{grantAuto}~p~app/ok}{s'}$ \\ \\
    \textit{El sistema de permisos de Android garantiza que un otorgamiento automático de permisos peligrosos puede ocurrir solamente para aquellos permisos que pertenecen a grupos autorizados por el usuario.}
\end{prop}

Sin embargo, algunas incógnitas surgen al intentar formalizar en qué situaciones un grupo de
permisos se encuentra autorizado. Por ejemplo, observamos que existen estados válidos del sistema en
los que un permiso puede ser otorgado automáticamente a una aplicación, a pesar de que
\textbf{actualmente} no existan otros permisos de ese mismo grupo ya otorgados a la misma. Esta
situación podría alcanzarse con la siguiente secuencia de acciones:

\begin{enumerate}
    \item Una aplicación $A$ declara el permiso $P$ agrupado en el grupo $G$
    \item Una aplicación $B$ solicita el permiso $P$
    \item El usuario otorga el permiso $P$ a la aplicación $B$
    \item La aplicación $A$ se desinstala (y por lo tanto los permisos declarados por ella son
          eliminados)
    \item La aplicación $B$ puede otorgar automáticamente los permisos pertenecientes al grupo $G$
          (a pesar de que ya no cuenta con el permiso $P$)
\end{enumerate}

Es importante mencionar que este escenario no implica que existe un falla de seguridad en el
sistema. Podría ser una decisión tomada al diseñar la plataforma con la intención de evitar abrumar
al usuario con advertencias y cuadros de diálogos solicitando acciones. Sin embargo, esta decisión
no está clara y no es desambiguada en ningún lugar de la documentación. A continuación formalizamos
el escenario descripto:

\begin{prop} \label{section:formalization:property2} \mbox{} \\ \\
    $\exists (s:\AndroidState) (p:\Perm) (g:\PermGrp) (app:\AppId),$ $valid\_state(s) \land \\
        getPermissionLevel(p) = dangerous~ \land getPermissionGroup(p) = Some\ g\ \land$ $\neg
        (\exists (p': \Perm), p' \in getGrantedPermsApp(app,s)~ \land$ \\
    $ getPermissionGroup(p') = Some\ g) \land Pre(s, grantAuto~p~a) \\ \\$

    \textit{El sistema puede otorgar automáticamente un permiso a pesar de que no hay otro permiso del mismo grupo actualmente otorgado a la aplicación.}
\end{prop}


La siguiente propiedad formaliza el escenario mencionado en la sección
\ref{subsection:recent-changes:grouped-permissions} sobre los permisos normales y peligrosos
compartiendo el  mismo grupo. Como mencionamos previamente, formalizamos el peor escenario posible
que satisface la especificación informal de la plataforma. Sin embargo, nuestra postura es que no
debería permitirse que permisos de distintos niveles de protección compartan grupo ya que podría
facilitar un ataque por escalamiento de privilegios. Por ejemplo, si un permiso $A$ con nivel de
protección normal comparte grupo con un permiso peligroso $B$, una aplicación podría obtener
autorización para que $B$ sea otorgado automáticamente en el momento en el que el sistema concede a
$A$ (es decir, en tiempo de instalación). Independientemente de si en una implementación real del
sistema de permisos el usuario es notificado de esta decisión, creemos que es una situación no
deseada dado que de alguna manera rompe con la idea de que los permisos peligrosos son otorgados en
tiempo de ejecución.


\begin{prop} \label{section:formalization:property3} \mbox{} \\ \\
    $\forall (s,s': \AndroidState)~(a: \AppId)~(m: \Manifest)~(c: \Cert)~(resources:
        list~\Res)~(g:\PermGrp)\\
        (pDang~pNorm: \Perm), \Mathexecrel{s}{\texttt{install}~a~m~c~resources/ok}{s'} \rightarrow$
    \\
    $getPermissionLevel(pDang) = dangerous~ \rightarrow \\
        getPermissionGroup(pDang) = Some\ g \rightarrow \\
        getPermissionLevel(pNorm) = normal~ \rightarrow \\
        getPermissionGroup(pNorm) = Some\ g \rightarrow \\
        \{pDang,~pNorm\} \subseteq getAppRequestedPerms(m) \rightarrow \\
        Pre(s', grantAuto~pDang~a)$ \\ \\

    \textit{Una aplicación que usa un permiso normal y uno peligroso del mismo grupo de permisos, puede obtener el segundo automáticamente luego de ser instalada.}
\end{prop}

Como mencionamos previamente, los usuarios tienen la posibilidad de revocar cualquier permiso
previamente otorgado en cualquier momento. Sin embargo, en el caso de los permisos agrupados, no es
posible hacerlo de manera granular. El grupo entero debe ser invalidado. Creemos que esta situación
es deseable, considerando que al otorgar un permiso agrupado el sistema concede cierto privilegio
sobre el grupo entero (en lugar del permiso solicitado en sí). Por lo tanto, probamos que nuestro
sistema es consistente con este comportamiento.

\begin{prop} \label{section:formalization:property4} \mbox{} \\ \\
    $\forall (s,s': \AndroidState)~(g:\PermGrp)~(app:\AppId),$ \\
    $\Mathexecrel{s}{\texttt{revokePermGroup}~g~app/ok}{s'} \rightarrow$ \\
    $\neg (\exists (p: \Perm), p \in getGrantedPermsApp(app,s')~ \\
    \land getPermissionGroup(p) = Some\ g) \\ \\$

    \textit{Cuando un usuario revoca el acceso a un grupo de permisos para determinada aplicación, todos los permisos individuales son revocados también.}
\end{prop}

El último cambio incorporado al modelo fue el que mencionamos en la sección
\ref{subsection:recent-changes:legacy-apps}. El mismo agrega restricciones a las acciones que pueden
ser ejecutadas por aplicaciones orientadas a versiones viejas de la plataforma, dado que las mismas
consiguieron todos los permisos en tiempo de instalación. La siguiente propiedad establece que
ninguna aplicación \textit{legacy} que \textbf{no ha sido verificada aún} puede recibir
\textit{intents}. De esta manera, ninguna de esas aplicaciones estará en condiciones de iniciar
nuevas actividades o servicios maliciosos con los permisos adquiridos en la instalación.

\begin{prop} \label{section:formalization:property5} \mbox{} \\ \\
    $\forall (s,s': \AndroidState)~(i:\Intent)~(ic:\iComp)~(app:\AppId),$ \\
    $oldAppNotVerified(app, s) \rightarrow$ \\
    $\neg \Mathexecrel{s}{\texttt{receiveIntent}~i~ic~app/ok}{s'}$ \\ \\

    \textit{Una aplicación vieja que no ha sido verificada por el usuario no está autorizada a recibir intents.}
\end{prop}


Finalmente, incluimos una propiedad que ha estado vigente en el modelo desde que fue actualizado
para incluir los cambios de la versión 6 de Android. Esta propiedad sigue siendo válida luego de la
actualización del modelo. Cualquier aplicación que desee enviar información por internet deberá
contar con un permiso llamado \texttt{INTERNET}. Sin embargo, como el nivel de protección del mismo
es ``normal'', basta con listarlo en la sección de permisos usados del manifiesto para conseguirlo.
Una vez más, criticamos esto dado que facilita escenarios de fuga de información. La propiedad
siguiente formaliza este comportamiento, presentando un argumento razonable para volver atrás este
cambio introducido en Android Marshmallow.

\begin{prop} \label{section:formalization:property6} \mbox{} \\ \\
    $ \forall (s:\AndroidState) (sac:SACall) (c:\Comp) (ic:\iComp) (p:\Perm), \\
        valid\_state(s) \rightarrow permSAC(p, sac) \rightarrow \\
        getPermissionLevel(p) = normal \rightarrow getPermissionId(p) \in \\
        getAppRequestedPerms(getManifestForApp(getAppFromCmp(c,s),s))\\
        \rightarrow (ic, c) \in getRunningComponents(s) \rightarrow \Mathexecrel{s}{\texttt{call}~ic~sac/ok}{s}$
    \\ \\

    \textit{Si la ejecución de un llamado a la API de Android solo requiere permisos con nivel de protección normal,
        basta con listar dicho permiso en el manifiesto para estar habilitado a realizar el llamado.}
\end{prop}

Más propiedades del modelo pueden encontrarse en la especificación completa~\cite{github-code}. No
fueron incluidas en este informe dado que no están estrictamente relacionadas a los cambios
introducidos por las últimas versiones de la plataforma. Sin embargo, todas las propiedades aún
presentes en la especificación han sido ratificadas, demostrando de esta manera que dichos
comportamientos se mantuvieron durante las sucesivas evoluciones de Android.