\chapter{Formalización del sistema de permisos}
\label{chapter:formalization}

En este capítulo describiremos la extensión que realizamos sobre una formalización
previa\cite{luna-cleiej} del sistema de permisos de la plataforma para modelar los nuevos
comportamientos que se introdujeron con las versiones 7, 8, 9 y 10 del sistema operativo. También
presentaremos las propiedades más importantes que extrajimos del modelo, focalizándonos en aquellas
nuevas o en las que se han visto considerablemente afectadas en las nuevas versiones.

Nuestra formalización del modelo de permisos puede pensarse como una máquina de estados (donde
guardamos la información propia del sistema y los permisos) que puede mutar a través de acciones cuya
semántica está dada en términos de pre y post condición. Profundizaremos sobre esta decisión en las
siguientes secciones.

\section{Lenguaje formal utilizado}
\label{section:formalization:formal-language}
La especificación del sistema se realizó dentro del framework de trabajo Coq~\cite{coq}. Coq es un
asistente de pruebas interactivo, que permite el desarrollo de programas consistentes con su especificación.
Para lograrlo, provee tres aspectos fundamentales:
\begin{enumerate}
    \item Un lenguaje de especificación que permite escribir expresiones lógicas de alto orden, algoritmos y teoremas;
    \item Un asistente de pruebas que permite desarrollar pruebas matemáticas verificadas;
    \item Una herramienta de extracción de programas, que permite sintetizar programas en lenguajes
    como OCaml~\cite{ocaml} o Haskell~\cite{haskell} a partir de las especificaciones formales escritas
    previamente. Los programas construidos de esta manera suelen llamarse \textit{programas
    certificados}. 
\end{enumerate}

El lenguaje lógico subyacente usado por Coq es el Calculo de Construcciones Inductivas~\cite{cic}
(también conocido como CIC, por sus siglas en inglés)


