\chapter{Formalización del sistema de permisos}
\label{chapter:formalization}

En este capítulo describiremos la extensión que realizamos sobre una formalización
previa\cite{luna-cleiej} del sistema de permisos de la plataforma para modelar los nuevos
comportamientos que se introdujeron con las versiones 7, 8, 9 y 10 del sistema operativo. También
presentaremos las propiedades más importantes que extrajimos del modelo, focalizándonos en aquellas
nuevas o en las que se han visto considerablemente afectadas en las nuevas versiones.


\section{Lenguaje formal utilizado}
\label{section:formalization:formal-language}
La especificación del sistema se realizó dentro del framework de trabajo Coq~\cite{coq}. Coq es un
asistente de pruebas interactivo, que permite el desarrollo de programas consistentes con su especificación.
Para lograrlo, provee tres aspectos fundamentales:
\begin{enumerate}
    \item Un lenguaje de especificación que permite escribir expresiones lógicas de alto orden, algoritmos y teoremas;
    \item Un asistente de pruebas que permite desarrollar pruebas matemáticas verificadas;
    \item Una herramienta de extracción de programas, que permite sintetizar programas en lenguajes
    como OCaml~\cite{ocaml} o Haskell~\cite{haskell} a partir de las especificaciones formales escritas
    previamente. Los programas construidos de esta manera suelen llamarse \textit{programas
    certificados}. 
\end{enumerate}

El lenguaje lógico subyacente usado por Coq es el Cálculo de Construcciones Inductivas~\cite{cic}
(también conocido como CIC, por sus siglas en inglés).

\section{Notación}
En las siguientes secciones  utilizaremos la misma notación de los trabajos
previos~\cite{luna-cleiej,betarte-2017,betarte-2016}. La misma es similar a la sintaxis de Haskell.

\subsection{Estructuras de datos y tipos generales}
TODO: COMPLETAR ESTA SECCIÓN

Los diccionarios (más conocidos como \textit{records} por su nombre en inglés) tendrán la forma $\{
l_1: T_1, ..., l_n: T_n \}$ y notaremos el acceso a cada uno de sus elementos como $r.l_n$. También
usaremos $\{ T \}$ para definir el tipo del conjunto que contiene elementos de tipo $T$. 

Definir tipos inductivos blablabla

Para denotar el tipo de las funciones utilizaremos el símbolo $ \rightarrow$. Por ejemplo, una función
$f$ que toma un argumento de tipo $A$ y uno de tipo $B$ y devuelve un valor de tipo $C$, se notará
como $F: A \rightarrow B \rightarrow C$. Usaremos siempre la notación currificada.

\subsection{Tipos básicos de Coq}
A continuación detallaremos algunos de los tipos básicos de Coq más utilizados en esta tesina.
\subsubsection*{Tipo \textit{option}} 
\begin{flalign*}
    Inductive\ option\ T\ &:= &&\\
    &|\ Some\ :\ T \rightarrow\ option\ T&&\\
    &|\ None\ :\ option\ T &&
\end{flalign*}
    

El tipo \texttt{option}, análogo a la mónada \texttt{Maybe} de Haskell\cite{maybe-haskell}, sirve para
representar la posible ausencia de valores. Dado que en Coq no es posible definir funciones
parciales, utilizaremos este tipo para totalizar a las mismas. De esta manera, una función parcial $g$
de $A$ en $B$, será definida como $g:\ A\ \rightarrow\ option\ B$. 

\subsubsection*{Tipo \textit{list}}
\begin{flalign*}
    Inductive\ list\ T\ &:= &&\\
    &|\ \texttt{(::)}\ :\ T \rightarrow\ list\ T\ \rightarrow\ list\ T\ &&\\
    &|\ nil\ :\ list\ T &&
\end{flalign*}

El tipo utilizado para denotar una lista de elementos. Está definido de manera inductiva, siendo
\texttt{::} un operador de punto fijo. 

\subsubsection*{Tipo \textit{nat}}
Utilizado para denotar a los números naturales.

\subsubsection*{Tipo \textit{Prop}}

Este tipo representa uno de los dos "universos" donde viven las proposiciones lógicas en Coq. El tipo
$Prop$ es impredicativo y por lo tanto, una proposición de este tipo no contiene valor
computacional\cite{proof-irrelevance} y puede ser descartada a la hora de extraer un programa.
Utilizaremos este tipo para escribir teoremas que nos permitan razonar sobre el modelo. El otro
universo que contiene proposiciones lógicas se llama $Set$ y las proposiciones de este tipo sí tienen
valor computacional y deben ser preservadas en los programas extraídos.


\section{Formalización de los componentes básicos}


\section{Definición de estado}
Nuestra formalización del sistema de permisos de Android puede pensarse como una máquina de estado
abstracta [CITA?]. En este modelo, los estados del sistema están conformados por dos componentes: uno
que almacena la información dinámica del sistema, como las aplicaciones instaladas y los permisos otorgados
a las mismas; y otro que contiene información estática como el manifiesto de cada aplicación.

\begin{align}
    \text{Insertar definición formal del estado}
\end{align}

El primer componente está conformado por los siguientes elementos:

\begin{itemize}
    \item Una lista con los identificadores de las aplicaciones instaladas,
    \item Una lista  información de las aplicaiones \textit{legacy} que ya han
        sido verificadas por el usuario. Es un subconjunto de \texttt{InstApps}.
    \item \texttt{}
\end{itemize}


\section{Acciones}

\section{Ejecuciones}

\section{Propiedades del modelo}