\chapter{Formalización del sistema de permisos}
\label{chapter:formalization}

En este capítulo describiremos la extensión que realizamos sobre una formalización
previa\cite{luna-cleiej} del sistema de permisos de la plataforma para modelar los nuevos
comportamientos que se introdujeron con las versiones 7, 8, 9 y 10 del sistema operativo. También
presentaremos las propiedades más importantes que extrajimos del modelo, focalizándonos en aquellas
nuevas o en las que se han visto considerablemente afectadas en las nuevas versiones.


\section{Lenguaje formal utilizado}
\label{section:formalization:formal-language}
La especificación del sistema se realizó dentro del framework de trabajo Coq~\cite{coq}. Coq es un
asistente de pruebas interactivo, que permite el desarrollo de programas consistentes con su especificación.
Para lograrlo, provee tres aspectos fundamentales:
\begin{enumerate}
    \item Un lenguaje de especificación que permite escribir expresiones lógicas de alto orden, algoritmos y teoremas;
    \item Un asistente de pruebas que permite desarrollar pruebas matemáticas verificadas;
    \item Una herramienta de extracción de programas, que permite sintetizar programas en lenguajes
    como OCaml~\cite{ocaml} o Haskell~\cite{haskell} a partir de las especificaciones formales escritas
    previamente. Los programas construidos de esta manera suelen llamarse \textit{programas
    certificados}. 
\end{enumerate}

El lenguaje lógico subyacente usado por Coq es el Cálculo de Construcciones Inductivas~\cite{cic}
(también conocido como CIC, por sus siglas en inglés).

\section{Notación}
En las siguientes secciones  utilizaremos la misma notación de los trabajos
previos~\cite{luna-cleiej,betarte-2017,betarte-2016}. La misma es similar a la sintaxis de Haskell.

\subsection{Estructuras de datos y tipos generales}
TODO: COMPLETAR ESTA SECCIÓN

Los diccionarios (más conocidos como \textit{records} por su nombre en inglés) tendrán la forma $\{
l_1: T_1, ..., l_n: T_n \}$ y notaremos el acceso a cada uno de sus elementos como $r.l_n$. También
usaremos $\{ T \}$ para definir el tipo del conjunto que contiene elementos de tipo $T$. 

Definir tipos inductivos blablabla

Para denotar el tipo de las funciones utilizaremos el símbolo $ \rightarrow$. Por ejemplo, una función
$f$ que toma un argumento de tipo $A$ y uno de tipo $B$ y devuelve un valor de tipo $C$, se notará
como $F: A \rightarrow B \rightarrow C$. Usaremos siempre la notación currificada.

\subsection{Tipos básicos de Coq}
A continuación detallaremos algunos de los tipos básicos de Coq más utilizados en esta tesina.
\subsubsection*{Tipo \textit{option}} 
\begin{flalign*}
    Inductive\ option\ T\ &:= &&\\
    &|\ Some\ :\ T \rightarrow\ option\ T&&\\
    &|\ None\ :\ option\ T &&
\end{flalign*}
    

El tipo \texttt{option}, análogo a la mónada \texttt{Maybe} de Haskell\cite{maybe-haskell}, sirve para
representar la posible ausencia de valores. Dado que en Coq no es posible definir funciones
parciales, utilizaremos este tipo para totalizar a las mismas. De esta manera, una función parcial $g$
de $A$ en $B$, será definida como $g:\ A\ \rightarrow\ option\ B$. 

\subsubsection*{Tipo \textit{list}}
\begin{flalign*}
    Inductive\ list\ T\ &:= &&\\
    &|\ \texttt{(::)}\ :\ T \rightarrow\ list\ T\ \rightarrow\ list\ T\ &&\\
    &|\ nil\ :\ list\ T &&
\end{flalign*}

El tipo utilizado para denotar una lista de elementos. Está definido de manera inductiva, siendo
\texttt{::} un operador de punto fijo. 

\subsubsection*{Tipo \textit{nat}}
Utilizado para denotar a los números naturales.

\subsubsection*{Tipo \textit{Prop}}

Este tipo representa uno de los dos "universos" donde viven las proposiciones lógicas en Coq. El tipo
$Prop$ es impredicativo y por lo tanto, una proposición de este tipo no contiene valor
computacional\cite{proof-irrelevance} y puede ser descartada a la hora de extraer un programa.
Utilizaremos este tipo para escribir teoremas que nos permitan razonar sobre el modelo. El otro
universo que contiene proposiciones lógicas se llama $Set$ y las proposiciones de este tipo sí tienen
valor computacional y deben ser preservadas en los programas extraídos.


\section{Estado del sistema}
\label{section:formalization:state}

\subsection{Formalización de los componentes básicos}

\subsection{Definición de estado}
Nuestra formalización del sistema de permisos de Android puede pensarse como una máquina de estado
abstracta [CITA?]. En este modelo, los estados del sistema están conformados por dos componentes: uno
que almacena la información dinámica del sistema, como las aplicaciones instaladas y los permisos otorgados
a las mismas; y otro que contiene información estática como el manifiesto de cada aplicación. Formalmente:

\begin{align*}
System\ :=\ \{ state: State;\ environment: Environment \}
\end{align*}


El componente $State$ está conformado por los siguientes elementos:

\begin{itemize}
    \item Una lista con los identificadores de las aplicaciones instaladas,
    \item Una lista  información de las aplicaciones \textit{legacy} que ya han sido verificadas por
    el usuario. Es un subconjunto de las aplicaciones inst.
    \item Un mapeo entre las aplicaciones instaladas y los permisos otorgados a cada una de ellas.
    \item Un mapeo entre las aplicaciones instaladas y los grupos de permisos para los cuales el
    usuario ha permitido el otorgamiento automático de permisos. 
    \item Un mapeo entre los componentes que define una aplicación y las instancias en ejecución de
    ellos.
    \item Un mapeo entre los recursos de un proveedor de contenidos para una aplicación y los permisos
    permantentes otorgados sobre el mismo.
    \item Un mapeo entre los recursos de un proveedor de contenidos para una aplicación y los permisos
    temporales otorgados sobre el mismo.
    \item Un mapeo indicando el valor de cada uno de los recursos de una aplicación.
    \item Una lista de los intents que han sido enviados, junto con su emisor.
\end{itemize}

Por otro lado, el componente $Environment$ contiene la siguiente información:

\begin{itemize}
    \item Un mapeo que asocia a cada aplicación con su archivo de manifiesto
    \item Un mapeo que asocia a cada aplicación con el certificado que se utilizó para firmarla
    \item Un mapeo que asocia a las aplicaciones con los permisos definidos por ellas
    \item Una lista de las aplicaciones pre-instaladas del sistema
\end{itemize}

A continuación daremos la definición formal de ambos componentes. El orden en el que se definen los
campos de cada componente es el mismo que el de las numeraciones previas.
\begin{flalign*}
State\ &:=\ \{ &&\\
       &apps:\ list\ idApp; \\
       &alreadyVerified:\ list\ idApp; \\
       &grantedPermGroups:\ mapping\ idApp\ (list\ idGrp); \\
       &perms:\ mapping\ idApp\ (list\ Perm); \\
       &running:\ mapping\ iCmp\ Cmp; \\
       &delPPerms:\ mapping\ (idApp\ *\ CProvider\ *\ uri)\ PType; \\
       &delTPerms:\ mapping\ (iCmp\ *\ CProvider\ *\ uri)\ PType; \\
       &resCont:\ mapping\ (idApp\ *\ res)\ Val; \\
       &sentIntents:\ list\ (iCmp*Intent) \\
\}
\end{flalign*}

\begin{flalign*}
Environment\ &:=\ \{ &&\\
       &manifest:\ mapping\ idApp\ Manifest; \\
       &cert:\ mapping\ idApp\ Cert; \\
       &defPerms:\ mapping\ idApp\ (list\ Perm); \\
       &systemImage:\ list\ SysImgApp; \\
\}
\end{flalign*}

De aquí en adelante, al hablar del estado del sistema, nos estaremos refiriendo al componente
$System$. En caso de que querramos referirnos a alguno de sus subcomponentes seremos explícitos con su
nombre en inglés.

\subsubsection{Estados válidos del sistema}
No todos los elementos que habitan el conjunto definido anteriormente son relevantes al sistema que
queremos estudiar. Por ejemplo, no queremos trabajar sobre un estado en el que una aplicación
pre-instalada del sistema y una aplicación instalada por el usuario puedan tener el mismo
identificador. 

Inicialmente, a la hora de definir los componentes, fue necesario pensar qué condiciones debían
cumplir estos estados para representar estados de Android que tengan sentido. En consecuencia,
definimos una noción de \textbf{estado válido} para restringir el universo de estados a aquellos que
cumplen ciertas condiciones que nos garantizarán que nuestros estados del modelo tienen sentido al
compararlos con estados reales del sistema. Vale aclarar, que nuestra definición de estado válido no
es completa y que, de alguna manera, está focalizada en las propiedades que probaremos luego.

Se definió formalmente un predicado $valid\_state$ que se satisface cuandos se cumplen las siguientes
condiciones:

\begin{itemize}
    \item Todos los componentes, ya sea que pertenezcan a una aplicación instalada por el usuario o a
    una aplicación pre-instalada en el sistema tienen identificadores diferentes.
    \item Ningún componente pertenece a más de una aplicación.
    \item Ningún componente en ejecución es una instancia de un proveedor de contenido (los mismos no se \textit{ejecutan}).
    \item Todo permiso temporalmente otorgado ha sido otorgado a un componente en ejecución y es sobre
    un recurso de un proveedor de contenido existente.
    \item Todo componente en ejecución pertenece a una aplicación instalada en el sistema.
    \item Toda aplicación que establece un valor a un recurso está instalada en el sistema.
    \item El dominio de las funciones parciales que definen el $manifest$, $cert$ y $defPerms$ es el
    conjunto de todas las aplicaciones instaladas por el usuario.
    \item El dominio de las funciones parciales que definen $grantedPermGroups$ y $perm$ es el
    conjunto de todas las aplicaciones del sistema, tanto instaladas por el usuario como
    pre-instaladas.
    \item Todas las aplicaciones del sistema tienen identificadores diferentes.
    \item Todos los permisos definidos por las aplicaciones tienen identificadores diferentes.
    \item Todos los permisos otorgados existen en el sistema.
    \item Todos los intents que han sido enviados tienen identificadores diferentes.
\end{itemize}

\section{Acciones}
Modelamos las operaciones de Android que nos interesan estudiar como un conjunto de acciones
(definidas a través del tipo \texttt{Action}), donde cada una de ellas determina la manera en la que
nuestro sistema puede transicionar.  La tabla~\ref{table:actions} resume todas las acciones
disponibles.

La semántica de las mismas está dada en términos de pre-condición y post-condición. Para ello,
definimos los predicados $Pre$ y $Post$ de manera tal que para que una acción $a$ pueda transicionar
el sistema desde un estado $s$ hacia otro estado $s'$, deberán cumplirse $Pre\ s\ a$ y $Post\ s\ s'\
a$. Notaremos la transición de un estado a otro de la siguiente manera: $s\ \step{a}\ s'$.

A continuación y a modo de ejemplo, describiremos informalmente las acciones que han sido introducidas
o modificadas con las últimas actualizaciones del sistema.

\subsubsection{Semántica de \texttt{grant}}

Esta operación es la encargada de otorgar un permiso $p$ a una aplicación $a$. La misma ya estaba
presente en la formalización de la que se partió en esta tesina. Sin embargo, su semántica ha sido
modificada a raíz de las actualizaciones de la plataforma mencionadas en la sección
\ref{subsection:recent-changes:grouped-permissions}. En particular, ahora el sistema podrá
transicionar con esta operación solo si  el permiso $p$ no pertenece a un grupo o, en caso de que
pertenezca, es el primero del grupo en ser otorgado a la aplicación. El resto de las precondiciones
necesarias para transicionar se mantuvieron: el permiso $p$ debe existir (es decir, debe estar
definido o bien por el sistema o por alguna aplicación), debe estar declarado como usado en el
manifiesto de la aplicación $a$, debe ser un permiso peligroso y no debe haber sido otorgado a la
aplicación previamente.

Si la precondición se cumple, el sistema transicionará hacia un estado en dónde el permiso se agrega a
los permisos otorgados a la aplicación, es decir, se agrega $p$ al conjunto $perms\ a$; y en caso de
corresponder, ocurre lo análogo con el grupo de $p$ y $grantedPermGroups\ a$. El resto de los
componentes del sistema no se verán modificados.

\subsubsection{Semántica de \texttt{grantAuto}}

Para modelar en su totalidad a los cambios mencionados en la sección
\ref{subsection:recent-changes:grouped-permissions}, además de los cambios en la semántica a la
operación \texttt{grant}, se introdujo una nueva acción \texttt{grantAuto}. Esta última difiere de la
primera en que el sistema solo podrá transicionar con ella si el permiso que se intenta otorgar
pertenece a un grupo que el usuario ya ha autorizado. En caso de que esto se cumpla, el modelo mutará
hacia un estado en donde la aplicación en cuestión obtuvo el permiso solicitado.

\subsubsection{Semántica de \texttt{revoke} y \texttt{revokePermGroup}}

De manera similar a lo ocurrido con \texttt{grant} y \texttt{grantAuto}, la semántica de las
operaciones  \texttt{revoke} y \texttt{revokeGroup} también se modificaron con los cambios de las
últimas versiones. Decidimos modelar estas operaciones de manera tal que se mantenga una relación con
la experiencia de usuario al revocar permsisos. De esta manera, la operación \texttt{revoke} es la
encargada de revocar permisos individuales \textbf{no} agrupados mientras que \texttt{revokePermGroup}
quita todos los permisos pertenecientes al grupo deseado. Nuestro modelo no permite revocar permisos
que pertenezcan a un grupo de manera individual.

Si el sistema transiciona con \texttt{revoke}, entonces dado un permiso $p$ y una aplicación $a$,
obtendremos un estado en donde la aplicación $a$ ya no tendrá acceso a los recursos protegidos por
$p$. Análogamente, dado un permiso $g$ y una aplicación $a'$, al transicionar con
\texttt{revokePermGroup}, la aplicación $a$ ya no tendrá ningún permiso perteneciente al grupo $g$.
Además, el sistema ya no estará autorizado a otorgar a la aplicación $a$ permisos del grupo $g$ de
manera automática.


\subsubsection{Semántica de \texttt{verifyOldApp}}

Esta operación se agregó al modelo para razonar sobre el nuevo el comportamiento mencionado en la
sección~\ref{subsection:recent-changes:legacy-apps}. Dada una aplicación $a$, para poder transicionar
el sistema utilizando la operación $verifyOldApp\ a$, deben cumplirse las siguientes condiciones: $a$
debe ser una aplicación instalada en el sistema, aún no debe haber sido ejecutada y debe estar
orientada a una versión previa a la sexta versión de Android.

En la implementación real de la plataforma, al momento de verificar una aplicación vieja se muestra al
usuario un menú con los permisos otorgados a la misma en el momento en que fue instalada, junto
con la posibilidad de elegir cuales de ellos se desea mantener y cuales revocar. Nuestra operación
$verifyOldApp\ a$ simplemente transiciona hacia un estado en donde a la aplicación $a$ se le han
revocado todos sus permisos y se ha marcado como verificada. Para modelar la acción en la que el
usuario selecciona los permisos que desea mantener mediante la interfaz ofrecida por Android,
deberemos dar una sucesión de acciones,  donde primero se verifica la aplicación y luego se otorgan
los permisos que se eligió mantener.

\subsubsection{Semántica de \texttt{receiveIntent}}

A raíz del cambio mencionado previamente y en la sección~\ref{subsection:recent-changes:legacy-apps},
agregamos una nueva condición que deberá cumplirse para que una aplicación pueda recibir un
\textit{intent}: para que una aplicación $a$ pueda recibir el intent $i$, entonces la misma no debe
ser una aplicación \textit{legacy} o, en caso de serlo, debe haber sido previamente verificada por el
usuario.

\begin{table}
    \label{table:actions}
    \vspace{0.2cm}
    \begin{tabularx}{\linewidth}{|l X|}
        \hline
        \textbf{Acción}   & \textbf{Descripción}                                                                                                                                               \\
        \hline
        $\mathtt{install}~app~m~c~res$     & Instala la aplicación con identificador $app$, cuyo manifiesto es $m$, su certificado es $c$ y la lista de recursos es $res$.                                                                                        \\
        \hline
        $\mathtt{uninstall}~app$           & Desinstala la aplicación con identificador $app$.                                                                                                                                                                    \\
        \hline
        $\mathtt{read}~ic~cp~u$            & El componente en ejecución $ic$ lee el recurso correspondiente al identificador URI $u$ del proveedor de contenido $cp$.                                                                                             \\
        \hline
        $\mathtt{write}~ic~cp~u~val$       & El componente en ejecución $ic$ escribe el valor $val$ en el recurso correspondiente al identificador $u$ del proveedor de contenido $cp$.                                                                           \\
        \hline
        $\mathtt{startActivity}~i~ic$      & El componente en ejecución $ic$ solicita comenzar la actividad especificada por el intent $i$.                                                                                                                       \\
        \hline
        $\mathtt{startActivityRes}~i~n~ic$ & El componente en ejecución $ic$ solicita comenzar la actividad especificada por el intent $i$ y espera como respuesta un token $n$.                                                                                  \\
        \hline
        $\mathtt{startService}~i~ic$       & El componente en ejecución $ic$ solicita comenzar el servicio especificado por el intent $i$.                                                                                                                        \\
        \hline
        $\mathtt{sendBroadcast}~i~ic~p$    & El componente en ejecución $ic$ envía el intent $i$ en modo \textit{broadcast}, especificando que solo los componentes que posean el permiso $p$ pueden recibirlo.                                                   \\
        \hline
        $\mathtt{sendOrdBroadcast}~i~ic~p$ & El componente en ejecución $ic$ envía el intent $i$ en modo \textit{broadcast} ordenado,  especificando que solo los componentes que posean el permiso $p$ pueden recibirlo.                                         \\
        \hline
        $\mathtt{sendSBroadcast}~i~ic$     & El componente en ejecución $ic$ envía el intent $i$ en modo \textit{sticky broadcast}.                                                                                                                               \\
        \hline
        $\mathtt{resolveIntent}~i~app$     & La aplicación $app$ vuelve al intent $i$ explícito.                                                                                                                                                                  \\
        \hline
        $\mathtt{stop}~ic$                 & El componente en ejecución $ic$ termina su ejecución.                                                                                                                                                                \\
        \hline
        $\mathtt{grantP}~ic~cp~app~u~op$   & El componente en ejecución $ic$ delega permisos permantentes a la aplicación $app$. Esta delegación autoriza a $app$ a realizar la operación $op$ en el recurso asignado al URI $u$ del proveedor de contenido $cp$. \\
        \hline
        $\mathtt{revokeDel}~ic~cp~u~op$    & El componente en ejecución $ic$ revoca los permisos otorgados al recurso $u$ del proveedor de contenidos $cp$ para realizar la operación $op$.                                                                       \\
        \hline
        $\mathtt{call}~ic~sac$             & El componente en ejecución $ic$ realiza el llamado a una función del sistema denominada $sac$.                                                                                                                       \\
        \hline
        $\mathtt{grant}~p~app$             & Otorga el permiso $p$ a la aplicación $app$ con la confirmación del usuario.                                                                                                                                         \\
        \hline
        $\mathtt{grantAuto}~p~app$         & Otorga automáticamente el permiso $p$ a la aplicación $app$ (sin requerir confirmación del usuario).                                                                                                                 \\
        \hline
        $\mathtt{revoke}~p~app$            & Revoca un permiso no agrupado $p$ de la aplicación $app$.                                                                                                                                                            \\
        \hline
        $\mathtt{revokePermGroup}~g~app$   & Revoca todos los permisos pertenecientes al grupo $g$ de la aplicación $app$.                                                                                                                                        \\
        \hline
        $\mathtt{hasPermission}~p~app$     & Chequea si la aplicación $app$ posee el permiso $p$.                                                                                                                                                                 \\
        \hline
        $\mathtt{receiveIntent}~i~ic~app$  & La aplicación $app$ recibe el intent $i$, enviado por el componente en ejecución $ic$.                                                                                                                               \\
        \hline
        $\mathtt{verifyOldApp}~app$        & El usuario verifica los permisos que han sido otorgados a la aplicación $app$. Solo se utiliza para aquellas aplicaciones que fueron instaladas previamente a la versión 6 de Android.                               \\
        \hline
    \end{tabularx}
    \caption{Acciones del sistema}
\end{table}

\section{Ejecuciones}
Cuando el sistema intenta ejecutar una acción $a$ en un estado válido $s$, hay dos posibles
resultados. Si la precondición de la acción se cumple, el sistema transicionará hacia otro estado $s'$
donde la postcondición de $a$ también se satisface. Sin embargo, si la precondición no se cumple, el
sistema permanecerá en el mismo estado en el que se encontraba al intentar la ejecución de $a$ y
responderá con un mensaje de error determinado por la relación $ErrorMsg$, definida a continuación.
Dados el estado $s$ y la aplicación $a$ mencionados previamente, y un código de error $ec$, la
relación $ErrorMsg s a ec$ se satisface si y sólo si el código $ec$ es una respuesta aceptable cuando
el sistema falla al ejecutar $a$ en el estado $s$.

Formalmente, las posibles respuestas de sistema se definen a través de la siguiente semántica operacional:
\begin{displaymath}
\begin{array}{c}
\inference[]{$$valid\_state(s)$$ \hspace{.2cm} $$Pre(s, a)$$ \hspace{.2cm} $$Post(s, a, s')$$}{$$s\step{a/ok}s'$$}
\hspace{0.5cm}
\inference[]{$$valid\_state(s)$$ \hspace{.2cm} $$ErrorMsg(s, a, ec)$$}{$$s\step{a/error(ec)}s$$}
\end{array}
\end{displaymath}

\section{Propiedades del modelo}

En esta sección presentaremos y discutiremos las propiedades que establecimos y demostramos sobre
nuestra formalización de Android. Nos enfocamos en propiedades de \textit{safety}\footnote{Utilizamos
el término en inglés para diferenciarlo de de \textit{security}, dado que en la traducción al español
mantener esa diferencia es más complejo.} aunque también formalizamos algunos comportamientos
potencialmente peligrosos que no han sido considerados en la especificación informal de la plataforma.

En la tabla \ref{table:auxiliary_functions} introducimos algunas funciones y predicados auxiliares que
nos ayudarán a definir los teoremas presentados.

\begin{table}[thb!]
    \centering
    \begin{tabularx}{\linewidth}{|l X|}
        \hline
        \textbf{Función/Predicado}   & \textbf{Descripción}                                                                                                                                               \\
        \hline
        $getAppFromCmp(c,s)$         & Dado un componente $c$ y un estado $s$, devuelve la aplicación a la cual pertenece dicho componente.                                                               \\
        \hline
        $getAppRequestedPerms(m)$    & Dado un manifiesto $m$ de una aplicación, devuelve los permisos listados como usados.                                                                              \\
        \hline
        $getGrantedPermsApp(app,s)$  & Devuelve los permisos con los que cuenta la aplicación $app$ en el estado $s$.                                                                                     \\
        \hline
        $getAuthorizedGroups(app,s)$ & Dada una aplicación $app$ y un estado $s$, devuelve los grupos de permisos que se encuentran autorizados para otorgar permisos automáticamente a dicha aplicación. \\
        \hline
        $getManifestForApp(app,s)$   & Devuelve el manifiesto de la aplicación $app$. El estado es necesario
        como argumento porque el manifiesto se encuentra guardado en la parte estática del mismo.                                                                                                         \\
        \hline
        $getPermissionId(p)$         & Devuelve el identificador del permiso $p$.                                                                                                                         \\
        \hline
        $getPermissionLevel(p)$      & Devuelve el nivel de protección del permiso $p$.                                                                                                                   \\
        \hline
        $getPermissionGroup(p)$      & Devuelve $Some~g$ si el permiso $p$ pertenece al grupo $g$, o $None$ en caso contrario.                                                                            \\
        \hline
        $getRunningComponents(s)$    & Devuelve un conjunto de pares conformados por el ID de una instancia en ejecución con el componente asociado a la misma.                                           \\
        \hline
        $oldAppNotVerified(app,s)$   & Válido si y solo si la aplicación $app$ es considerada $legacy$ y el usuario aún no la ha verificado en el estado $s$.                                               \\
        \hline
    \end{tabularx}
    \caption{Funciones auxiliares y predicados}
    \label{table:auxiliary_functions}
\end{table}


A raíz del cambio mencionado en la sección \ref{subsection:recent-changes:grouped-permissions}, la
primer propiedad que formulamos establece una condición necesaria para que nuestra formalización del
sistema sea consecuente con la documentación de la plataforma: solo los permisos que pertenecen a un
grupo ya autorizado por el usuario pueden ser otorgados de manera automática por el sistema.

\begin{prop} \label{section:formalization:property1}
   \mbox{} \\ \\
   $\forall (s,s': \AndroidState) (p:\Perm) (g:\PermGrp) (app:\AppId),$ \\
   $getPermissionLevel(p) = dangerous \land getPermissionGroup(p) = Some\ g\ \land
   g \notin getAuthorizedGroups(app,s)
   \rightarrow\ \neg \Mathexecrel{s}{\texttt{grantAuto}~p~app/ok}{s'}$ \\ \\
   \textit{El sistema de permisos de Android garantiza que un otorgamiento automático de permisos peligrosos puede ocurrir solamente para aquellos permisos que pertenecen a grupos autorizados por el usuario.}
\end{prop}

Sin embargo, algunas incógnitas surgen al intentar formalizar en qué situaciones un grupo de permisos
se encuentra autorizado. Por ejemplo, observamos que existen estados válidos del sistema en los que un
permiso puede ser otorgado automáticamente a una aplicación, a pesar de que \textbf{actualmente} no
existan otros permisos de ese mismo grupo ya otorgados a la misma. Esta situación podría alcanzarse
con la siguiente secuencia de acciones:

\begin{enumerate}
    \item Una aplicación $A$ declara el permiso $P$ agrupado en el grupo $G$
    \item Una aplicación $B$ solicita el permiso $P$
    \item El usuario otorga el permiso $P$ a la aplicación $B$
    \item La aplicación $A$ se desinstala (y por lo tanto los permisos declarados por ella son eliminados)
    \item La aplicación $B$ puede otorgar automáticamente los permisos pertenecientes al grupo $G$ (a
    pesar de que ya no cuenta con el permiso $P$)
\end{enumerate}

Es importante mencionar que este escenario no implica que existe un falla de seguridad en el sistema.
Podría ser una decisión tomada al diseñar la plataforma con la intención de evitar abrumar al usuario
con advertencias y cuadros de diálogos solicitando acciones. Sin embargo, esta decisión no está clara
y no es desambiguada en ningún lugar de la documentación. A continuación formalizamos el
escenario descripto:

\begin{prop} \label{section:formalization:property2}
    \mbox{} \\ \\
   $\exists (s:\AndroidState) (p:\Perm) (g:\PermGrp) (app:\AppId),$
   $valid\_state(s) \land \\
   getPermissionLevel(p) = dangerous~ \land
   getPermissionGroup(p) = Some\ g\ \land$
   $\neg (\exists (p': \Perm), p' \in getGrantedPermsApp(app,s)~ \land$ \\
   $ getPermissionGroup(p') = Some\ g) \land Pre(s, grantAuto~p~a) \\ \\$

   \textit{El sistema puede otorgar automáticamente un permiso a pesar de que no hay otro permiso del mismo grupo actualmente otorgado a la aplicación.}
\end{prop}


La siguiente propiedad formaliza el escenario mencionado en la sección
\ref{subsection:recent-changes:grouped-permissions} sobre los permisos normales y peligrosos
compartiendo el  mismo grupo. Como mencionamos previamente, formalizamos el peor escenario posible que
satisface la especificación informal de la plataforma. Sin embargo, nuestra postura es que no debería
permitirse que los permisos de distintos niveles de protección ya que podría facilitar un ataque por
escalamiento de privilegios.

\begin{prop} \label{section:formalization:property3}
 \mbox{} \\ \\
$\forall (s,s': \AndroidState)~(a: \AppId)~(m: \Manifest)~(c: \Cert)~(resources: list~\Res)~(g:\PermGrp)\\
(pDang~pNorm: \Perm), \Mathexecrel{s}{\texttt{install}~a~m~c~resources/ok}{s'} \rightarrow$ \\
$getPermissionLevel(pDang) = dangerous~ \rightarrow \\
getPermissionGroup(pDang) = Some\ g \rightarrow \\
getPermissionLevel(pNorm) = normal~ \rightarrow \\
getPermissionGroup(pNorm) = Some\ g \rightarrow \\
\{pDang,~pNorm\} \subseteq getAppRequestedPerms(m) \rightarrow \\
Pre(s', grantAuto~pDang~a)$ \\ \\

\textit{Una aplicación que usa un permiso normal y uno peligroso del mismo grupo de permisos, puede obtener el segundo automáticamente luego de ser instalada.}
\end{prop}

Los usuarios tienen la posibilidad de revocar cualquier permiso previamente otorgado en cualquier
momento. Sin embargo, en el caso de los permisos agrupados, no es posible hacerlo de manera granular.
El grupo entero debe ser invalidado. Creemos que esta situación es deseable, considerando que al
otorgar un permiso agrupado el sistema concede cierto privilegio sobre el grupo entero (en lugar del
permiso solicitado en sí). Por lo tanto, probamos que nuestro sistema es consistente con este
comportamiento.

\begin{prop} \label{section:formalization:property4}
 \mbox{} \\ \\
$\forall (s,s': \AndroidState)~(g:\PermGrp)~(app:\AppId),$ \\
$\Mathexecrel{s}{\texttt{revokePermGroup}~g~app/ok}{s'} \rightarrow$ \\
$\neg (\exists (p: \Perm), p \in getGrantedPermsApp(app,s')~ \\
\land getPermissionGroup(p) = Some\ g) \\ \\$

\textit{Cuando un usuario revoca el acceso a un grupo de permisos para determinada aplicación, todos los permisos individuales son revocados también.}
\end{prop}

El último cambio incorporado al modelo fue el que mencionamos en la sección
\ref{subsection:recent-changes:legacy-apps}. El mismo agrega restricciones a las acciones que pueden
ser ejecutadas por aplicaciones orientadas a versiones viejas de la plataforma, dado que las mismas
consiguieron todos los permisos en tiempo de instalación. La siguiente propiedad establece que ninguna
aplicación \textit{legacy} que \textbf{no ha sido verificada aún} puede recibir \textit{intents}. De
esta manera, ninguna de esas aplicaciones estará en condiciones de iniciar nuevas actividades o
o servicios maliciosos con los permisos adquiridos en la instalación.

\begin{prop} \label{section:formalization:property5}
 \mbox{} \\ \\
$\forall (s,s': \AndroidState)~(i:\Intent)~(ic:\iComp)~(app:\AppId),$ \\
$oldAppNotVerified(app, s) \rightarrow$ \\
$\neg \Mathexecrel{s}{\texttt{receiveIntent}~i~ic~app/ok}{s'}$ \\ \\

\textit{Una aplicación vieja que no ha sido verificada por el usuario no está autorizada a recibir intents.}
\end{prop}


Finalmente, incluimos una propiedad que ha estado vigente en el modelo desde que fue actualizado para
incluir los cambios de la versión 6 de Android. Cualquier aplicación que desee enviar información por
internet deberá contar con un permiso llamado \texttt{INTERNET}. Sin embargo, como el nivel de
protección del mismo es "normal", basta con listarlo en la sección de permisos usados del manifiesto
para conseguirlo. Una vez más, criticamos esto dado que facilita escenarios de fuga de información. La
propiedad siguiente formaliza este comportamiento, presentando un argumento razonable para volver
atrás este cambio introducido en Android Marshmallow.

\begin{prop} \label{section:formalization:property6}
 \mbox{} \\ \\
$ \forall (s:\AndroidState) (sac:SACall) (c:\Comp) (ic:\iComp) (p:\Perm), \\
valid\_state(s) \rightarrow permSAC(p, sac) \rightarrow \\
getPermissionLevel(p) = normal \rightarrow getPermissionId(p) \in \\
getAppRequestedPerms(getManifestForApp(getAppFromCmp(c,s),s)) \rightarrow\\
(ic, c) \in getRunningComponents(s) \rightarrow \Mathexecrel{s}{\texttt{call}~ic~sac/ok}{s}$ \\ \\

\textit{Si la ejecución de un llamado a la API de Android solo requiere permisos con nivel de protección normal, basta con listar dicho permiso en el manifiesto para estar habilitado a realizar el llamado.}
\end{prop}