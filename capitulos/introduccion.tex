\chapter{Introducción}

Android\cite{android-proyect} es uno de los sistemas operativos para celulares más utilizados en el
mundo, capturando aproximadamente el 85\% del mercado mundial\cite{idc-report}. Android también
posee una tienda oficial, accesible desde cualquier versión del sistema, que permite instalar un
sinfín de aplicaciones para facilitar tareas del usuario, muchas de ellas, críticas para la
privacidad. Por ejemplo, una aplicación para editar fotos probablemente solicite acceso a todas las
fotos de la galería para funcionar correctamente, o una aplicación que permita enviar mensajes de
texto posiblemente solicite tener acceso a los mensajes históricos del usuario.

Por lo tanto, para ofrecer las garantías de seguridad y privacidad que se necesitan en un sistema de
estas características, Android implementa un protocolo de consenso \textit{multi-parte}, donde una
acción ocurre sólo si todos los actores involucrados en la tarea están de acuerdo. Supongamos que un
usuario quiere editar sus fotos y decide utilizar una aplicación exclusiva para ese propósito. En
ese caso, necesitaremos el consenso de las siguientes partes:

\begin{itemize}
    \item \textbf{El usuario:} al elegir qué aplicación utilizará y al otorgar explícitamente los
          permisos que sean necesarios
    \item \textbf{Los desarrolladores:} al especificar los permisos que necesitan para que su
          aplicación pueda acceder a los recursos necesarios para funcionar correctamente
    \item \textbf{La plataforma:} al actuar como mediador entre las otras dos partes,
          supervisando que el pacto se cumpla y que solo los recursos protegidos bajo el permiso
          otorgado sea accesible por la aplicación.
\end{itemize}

El hecho de que el último actor sea, a fin de cuentas, una pieza de software, lo convierte en un
objetivo principal a la hora de aplicar métodos formales para su verficación. Estudiar formalmente
estas políticas de seguridad resulta fundamental para lograr un entendimiento preciso sobre qué es
lo que se quiere o espera del sistema y las garantías que el mismo está en condiciones de ofrecer.

La importancia de estudiar este tipo de mecanismos de seguridad fue señalada originalmente por el
reporte de Anderson~\cite{Anderson:1972}, donde el concepto de \textit{monitor de referencia} fue
introducido por primera vez. Este concepto define los requerimientos para el diseño de un
\textit{mecanismo de validación por referencia}, quien es el responsable de imponer las políticas de
control de acceso de un sistema. Para su correcto funcionamiento, Anderson definió tres
requerimientos de diseño:

\begin{itemize}
    \item \textbf{Mediación completa:} el mecanismo de validación por referencia (MVR) siempre debe
          ser invocado luego de ejecutar una acción.
    \item \textbf{A prueba de manipulaciones}, o más conocido como \textit{tamper-proof} en inglés:
          el MVR no debe ser modificable a mano o programáticamente para garantizar la
          integridad del mismo.
    \item \textbf{Verificable:} el MVR debe ser lo suficientemente pequeño para poder demostrar
          lógicamente que es completo y que su implementación es correcta.
\end{itemize}

El trabajo presentado en esta tesina estudia el tercer requerimiento. En particular, se concentra en
analizar formalmente y verificar propiedades sobre un modelo idealizado del sistema de permisos de
Android, donde no se tienen en cuenta los detalles de implementación. Sin embargo, aún así provee un
escenario realista para investigar sobre esta parte crítica del sistema operativo Android.

Esta tesina parte de un trabajo previo realizado por Felipe Gorostiaga\cite{fgorostiaga}, en el que
se presentó un modelo idealizado para el sistema de permisos de Android 6. En este trabajo, se
extiende dicho modelo con nuevas características introducidas en la versiones 7, 8, 9 y 10 de
Android. Algunos de los cambios que los desarrolladores de la plataforma introdujeron con estas
versiones no implicaron cambios sobre el modelo abstracto. Para esos casos, se incluye un pequeño
análisis informal sobre las mejoras de seguridad que implican para la plataforma. Además, extendimos
una implementación funcional previa del MVR para que contemple los nuevos cambios en la
especificación.

\paragraph{Organización del informe}
El capítulo \ref{chapter:background} introduce infromalemnte los conceptos necesarios para entender
el funcionamiento del sistema de permisos de Android. También incluye un análisis informal sobre
todos los cambios considerados en este trabajo. Los capítulos \ref{chapter:formalization} y
\ref{chapter:implementation} presentan la formalización del modelo idealizado y su implementación,
respectivamente. Al final de cada uno, se discuten propiedades de relevancia relacionadas a los
nuevos cambios. En el capítulo \ref{chapter:estado} se analizan trabajos similares que también
estudian el sistema de permisos de Android. Se incluyeron diferentes enfoques y técnicas, a pesar de
que no todos los trabajos citados son de índole formal. Por último, en el capítulo
\ref{chapter:conclusion} contiene una conclsuión e ideas sobre trabajo futuro.