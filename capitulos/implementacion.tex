\chapter{Implementación de un prototipo certificado}
\label{chapter:implementation}
Formalizar sistemas sensibles, como un sistema de permisos de una plataforma usada masivamente, es
una herramienta teórica muy importante. Permite establecer propiedades sobre el sistema en cuestión
y razonar formalmente sobre ellas. Mantener esa formalización actualizada, por su parte, otorga la
posibilidad de estudiar las características más recientes de la plataforma, y en muchas situaciones,
detectar comporatmientos erróneos o inseguros antes de que sean aprovechados maliciosamente.

Por otro lado, contar con una especificación formal de un sistema permite desarrollar
implementaciones \textit{certificadas}. Una implementación certificada, es una implementación que
cumple correctamente con una especificación y que dicha corrección ha sido demostrada
matemáticamente. Este trabajo además de extender la especificación del sistema de permisos de
Android, extiende la implementación de la plataforma desarrollada en los trabajos previos
\cite{betarte-2017,luna-cleiej}. Claro está, que al partir de un modelo tan abstracto como el
nuestro, la implementación obtenida no estaría en condiciones de sustituir a la implementación real
de Android. Sin embargo, sí podría utilizarse como un monitor de referencia. Explicaremos esto con
más detalles en la sección \ref{section:future-work:reference-monitor}.

A continuación, describiremos las principales características de nuestra implementación del sistema
de permisos de Android y algunas propiedades demostradas sobre la misma.

\section{Características principales}
Nuestra implementación de la plataforma consiste de un conjunto de funciones definidas en Coq de
manera tal que por cada acción dada en la especificación existe una función encargada de llevarla a
cabo. Estas funciones son, básicamente, transformadores de estado. Si bien sus definiciones son
distintas entre sí, pues tienen semánticas muy diferentes, todas respetan el mismo patrón. Primero,
en un estado inicial, se evalúa si una expresión \textit{booleana} equivalente a la precondición de
la acción se satisface. En caso de que la misma se satisfaga, se ejecuta una función auxiliar
encargada de mutar el estado de manera tal que el nuevo estado cumpla con la postcondición
establecida por la acción del modelo. En caso de que la precondición no se cumpla, el estado no
sufre cambios y se devuelve un mensaje de error.

A modo de ejemplo, explicaremos con un poco más de detalle la implementación de la acción $grant$.
En la figura \ref{fig:install_action} puede observarse su definición formal. La definición formal
completa del esta y el resto de las operaciones puede encontrarse en el anexo.
\todoGuido{Tiene sentido agregar un anexo para esto? Si no referencio directamente el código}

\begin{figure}[ht]
    \begin{displaymath}
        \begin{array}{l}
            \textbf{Definition}\ grant\_safe(perm, app, s)\ : Result\ :=                \\
            \quad \textbf{match}\ grant\_pre(perm, app, s)\ \textbf{with}               \\
            \quad \hspace{1cm} |\ Some\ ec\ \Rightarrow\ \{\ap{error}{ec},s\}           \\
            \quad \hspace{1cm} |\ None\ \Rightarrow\ \{ok, grant\_post(perm, app, s) \} \\
            \quad\ \textbf{end}.                                                        \\
        \end{array}
    \end{displaymath}
    \caption{Definición de la función $grant\_safe$, encargada de ejecutar la acción $grant$.}
    \label{fig:install_action}
\end{figure}

La función $grant\_pre$ se define como la conjunción de todas las condiciones establecidas por la
precondición de $grant$. En ella también se especifica el error que debe mostrarse cuando alguna de
ellas no se cumple. Por otro lado, la función $grant\_post$ implementa la modificación esperada en el
estado: el permiso $perm$ pasa a formar parte de los permisos otorgados a la aplicación $app$ y, si
el mismo estuviera agrupado, su grupo también se agregaría al listado de grupos autorizados por el
usuario para el otorgamiento automático de permisos a la aplicación correspondiente.
\todoGuido{en algún lugar de por acá tengo que mostrar el tipo Result}

Luego, tanto a $grant\_safe$ como al resto de las operaciones definidas de manera análoga, se las
agrupa en una función llamada $step$, que actúa principalmente como un despachador de acciones. En
la figura \ref{fig:step} se muestra la estructura de la misma.

\begin{figure}[ht]
    \begin{displaymath}
        \begin{array}{l}
            \textbf{Definition}\ step(s,a)\ :=\                                                               \\
            \quad \textbf{match}\ a\ \textbf{with}                                                            \\
            \quad  \hspace{1cm} |\ \ldots \Rightarrow\ \ldots                                                 \\
            \quad  \hspace{1cm} |\  \texttt{grant}\ perm\ app\ \Rightarrow\ grant\_safe(perm, app, s)         \\
            \quad  \hspace{1cm} |\  \texttt{grantAuto}\ perm\ app\ \Rightarrow\ grantAuto\_safe(perm, app, s) \\
            \quad  \hspace{1cm} |\ \ldots \Rightarrow\ \ldots                                                 \\
            \quad\ \textbf{end}.
        \end{array}
    \end{displaymath}
    \caption{Estructura de la función $step$}
    \label{fig:step}
\end{figure}

Finalmente, definimos una función $trace$ que captura la idea de ejecutar acciones sucesivas en el
sistema. Dado un estado inicial y una lista de acciones, esta función será la encargada de ir
concatenandolas de manera tal que el estado obtenido como resultado de una acción sea utilizado como
estado inicial de la siguiente. Mostramos su definición a continuación.

\begin{displaymath}
    \begin{array}{l}
        \textbf{Function}\ trace\ (s:\AndroidState)\ (actions:list\ \Action)\ :\ list\ \AndroidState \ :=                             \\
        \quad \textbf{match}\ actions\ \textbf{with}                                                                                  \\
        \quad \hspace{1cm} |\ nil\ \Rightarrow\ nil                                                                                   \\
        \quad \hspace{1cm} |\ action::rest\ \Rightarrow\ \textbf{let}\ s'\ :=\ (step\ s\ action).st\ \textbf{in}\ s'::trace\ s'\ rest \\
        \quad \textbf{end}.
    \end{array}
\end{displaymath}

\section{Corrección de la implementación}
Como mencionamos previamente, contar con una formalización de un sistema, permite desarrollar
sistemas correctos, confiables, en el sentido de tener garantías de que el comportamiento de los
mismos será adecuado según lo dictamine la especificación de cada uno. En este caso, demostramos un
teorema en Coq que garantiza que cualquier acción ejecutada por nuestra implementación, partiendo de
un estado válido, es un ejecución correcta del sistema.


\begin{theorem}
    [Corrección del monitor de referencia] \label{theorem:soundness}
    \mbox{} \\
    $ \begin{array}[t]{l}
            \forall\ (s:\AndroidState)\ (a:\Action),     \\
            \quad \quad \ap{valid\_state}{s} \rightarrow\
            s\step{a/step(s,a).resp} \bap{step}{s}{a}.st \\
        \end{array} $
\end{theorem}

Comenzamos la demostración de este teorema haciendo inducción en la acción a ejecutar. De esta
manera, podemos subdividir la prueba en lemas más pequeños, y demostrar la corrección de cada una de
las operaciones por separado. Luego, cada uno de estos sub-lemas posee la misma estructura: dada una
acción, demostraremos por un lado que la ejecución es correcta cuando se cumple la pre-condición; y
por otro, demostraremos que la ejecución también lo es cuando la pre-condición no se satisface. A
modo de ejemplo, presentamos el caso de la operación $grant$.

\begin{lemma}
    [Corrección de una ejecución válida de $grant$]
    \mbox{} \\
    $ \begin{array}[t]{l}
            \forall\ (s:\AndroidState)\ (app:\AppId)\ (p:\Perm),                          \\
            \quad \quad \ap{valid\_state}{s} \rightarrow\ Pre(s, grant\ p\ a) \rightarrow \\
            \quad \quad Post(                                                             \\
            \quad \quad \quad grant\_safe(p, app, s).st,\ grant\ p\ a)                    \\
        \end{array} $
\end{lemma}

\begin{lemma}
    [Corrección de una ejecución errónea de $grant$]
    \mbox{} \\
    $ \begin{array}[t]{l}
            \forall\ (s:\AndroidState)\ (app:\AppId)\ (p:\Perm),                                           \\
            \quad \quad \ap{valid\_state}{s} \rightarrow\ \neg Pre(s, grant\ p\ a) \rightarrow             \\
            \quad \quad grant\_safe(p, app, s).st = s\ \land                                               \\
            \quad \quad \quad (\exists\ (ec: \ErrorCode),\ grant\_safe(p,\ app,\ s).response\ =\ error\ ec \\
            \quad \quad \quad \quad \land\ ErrorMsg\ s\ (grant\ p\ a)\ ec)                                 \\
        \end{array} $
\end{lemma}

Con esos dos lemas demostrados, obtener una prueba de la corrección de la operación $grant$ se
reduce, a grandes rasgos, a diferenciar los casos en que la pre-condición se cumple de lo que no,
para luego aplicar el sub-lema correspondiente.

\begin{theorem}
    [Corrección de la operación $grant$] \label{lemma:soundness:grant}
    \mbox{} \\
    $ \begin{array}[t]{l}
            \forall\ (s:\AndroidState)\ (app:\AppId) (p:\Perm),                                                    \\
            \quad \quad \ap{valid\_state}{s} \rightarrow\                                                          \\
            \quad \quad \quad s\step{grant\ p\ a\ /\ step(s,\ grant\ p\ a).response} \bap{step}{s}{grant\ p\ a}.st \\
        \end{array} $
\end{theorem}

La demostración completa del lema \ref{lemma:soundness:grant} puede encontrarse en el módulo
\texttt{GrantIsSound.v} en el código de la formalización\cite{github-code}. Análogamente, la
demostración para cada una de las otras acciones puede encontrarse en el módulo
\texttt{<Action>IsSound.v}. La demostración del teorema \ref{theorem:soundness} se encuentra en el
módulo \texttt{Soundness.v}.

\section{Propiedades sobre la implementación}