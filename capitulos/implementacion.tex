\chapter{Implementación de un prototipo certificado}
\label{chapter:implementation}
Formalizar sistemas sensibles, como un sistema de permisos de una plataforma usada masivamente, es
una herramienta teórica muy importante. Permite establecer propiedades sobre el sistema en cuestión
y razonar formalmente sobre ellas. Mantener esa formalización actualizada, por su parte, otorga la
posibilidad de estudiar las características más recientes de la plataforma, y en muchas situaciones,
detectar comporatmientos erróneos o inseguros antes de que sean aprovechados maliciosamente.

Por otro lado, contar con una especificación formal de un sistema permite desarrollar
implementaciones \textit{certificadas}. Una implementación certificada, es una implementación que
cumple correctamente con una especificación y que dicha corrección ha sido demostrada
matemáticamente. Este trabajo además de extender la especificación del sistema de permisos de
Android, extiende la implementación de la plataforma desarrollada en los trabajos previos
\cite{betarte-2017,luna-cleiej}. Claro está, que al partir de un modelo tan abstracto como el
nuestro, la implementación obtenida no estaría en condiciones de sustituir a la implementación real
de Android. Sin embargo, sí podría utilizarse como un monitor de referencia. Explicaremos esto con
más detalles en la sección \ref{section:future-work:reference-monitor}.

A continuación, describiremos las principales características de nuestra implementación del sistema
de permisos de Android y algunas propiedades demostradas sobre la misma.

\section{Características principales}
