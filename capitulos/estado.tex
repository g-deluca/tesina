
\chapter*{Estado del arte}
\label{chapter:estado}
\addcontentsline{toc}{chapter}{\nameref{chapter:estado}}

% TODO: Párrafo introductorio
% Estadísticas de uso de celulares? Uso de Android? Estadísticas sobre ataques o
% ataques grandes conocidos?

Párrafo introductorio sobre el crecimiento del uso de celulares y de la cantidad
de aplicaciones. Mencionar el crecimiento de las aplicaciones, posibildad de
instalación de aplicaciones por fuera de las tiendas oficiales, y que las mismas
pueden utilizar recursos del sistema. Desde un punto de vista de seguridad y
privacidad, los recursos a los que pueden acceder las aplicaciones no son todos
iguales; por ejemplo, acceder a los contactos de alguien es más peligroso que
cambiar el fondo de pantalla del teléfono. Asimismo, no todas las aplicaciones
tienen la misma exigencia en cuanto a privacidad de sus datos, los
desarrolladores deben ser capaces de elegir qué datos de su aplicación compartir
y bajo qué condiciones.

Para obtener un balance entre las necesidades de los usuarios y de los
desarrolladores, Android se basa fundamentalmente en un \textit{sistema de
permisos} en el que al momento de realizar una acción, se tiene en cuenta a
todas las partes involucradas. En caso de que alguna de esas partes no esté de
acuerdo con la acción, ésta es bloqueada. Sin dudas, esto es algo novedoso si lo
comparamos con los modelos de seguridad implementados por otros sistemas
operativos, que están enfocados en el control de acceso y que autorizan al
ususario del sistema a ser quien tome decisión final sobre qué puede realizar una
aplicación.
%TODO: ¿Profundizo más en "¿Quiénes son todas las partes involucradas?"?
A lo largo de los últimos años, 