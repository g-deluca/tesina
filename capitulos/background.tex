\chapter{Análisis informal de Android}
\label{chapter:background}

\section{Arquitectura de la plataforma}
\label{section:architecture}

El sistema operativo Android está compuesto por cinco capas de software, ordenadas en forma de pila (o
\textit{stack}), donde cada una de ellas provee un grupo de servicios a la capa inmediatamente
superior. De esta forma, se va abstrayendo progresivamente la interacción con el hardware (la base de
la pila) hasta llegar al nivel más alto en el que se ubican las aplicaciones que realizaran las tareas
requeridas por los usuarios. A continuación, analizaremos brevemente cada uno de estos niveles:

\subsection{Núcleo del sistema operativo: Linux}
La base de la plataforma Android es el \textit{kernel} de Linux. Esto permite aprovechar diferentes
utilidades de un software ampliamente conocido y mantenido por la comunidad; como por ejemplo, la
generación de subprocesos y la administración de memoria de bajo nivel.

A nivel seguridad, Linux provee a Android ciertas características fundamentales para desarrollar el
sistema de permisos que se construye en las capas superiores. Esas características son: un sistema
multi-usuarios con permisos por usuario, aislamiento de procesos junto a un mecanismo extensible de
comunicación entre los mismos (también  conocido como IPC, por sus siglas en inglés:
\textit{Inter-process communication}) y el hecho de poder remover o deshabilitar partes innecesarias y
potencialmente inseguras del kernel.

Esto permite que los fabricantes de
dispositivos desarrollen controladores de hardware para un kernel ampliamente conocido



