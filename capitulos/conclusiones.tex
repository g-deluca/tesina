\chapter{Conclusiones y trabajo futuro}
\label{chapter:conclusion}

En esta tesina hemos actualizado una especificación formal previa del modelo de permisos de
Android\cite{luna-cleiej,betarte-2016}, agregando funcionalidades nuevas y modificando la semántica
de algunas operaciones ya existentes de acuerdo a los cambios introducidos en las versiones 7, 8, 9
y 10 de la plataforma. Con un acercamiento conservador, primero revisamos las propiedades ya
existentes del modelo para verificar si seguían siendo válidas. Algunas de ellas fueron eliminadas,
otras mínimamente modificadas por cambios en el modelo que no las afectaban directamente y otras han
cambiado radicalmente para adaptarse a las últimas modificaciones introducidas. También se han
agregado propiedades completamente nuevas. Entre las propiedades agregadas, se han incluido varias
cuyo principal objetivo es resaltar cómo los métodos formales pueden ayudar a desambiguar
comportamientos poco claros que pueden ser inferidos a partir de una especificación informal. En
particular, la propiedad \ref{section:formalization:property3} presenta un escenario en donde una
aplicación podría conseguir acceso a los permisos peligrosos de una aplicación sin consentimiento
explícito del usuario. Este escenario es posible dentro de la especificación informal descripta en
la documentación oficial de Android.

Además -y en consecuencia de haber actualizado la formalización- hemos actualizado la implementación
funcional del monitor de referencia con las nuevas características y hemos actualizado su prueba de
corrección. El código Coq en su totalidad puede encontrarse en Github\cite{github-code} y contiene
aproximadamente 23000 líneas de código, incluyendo las pruebas.

El objetivo más importante de esta tesina es tratar de mantener al día esta formalización del
sistema de permisos de Android para formar un \textit{framework} confiable que nos permita razonar y
establecer propiedades sobre el mismo. Utilizar modelos idealizados y prototipos certificados es
algo positivo \textit{per se}, que permite generar más transparencia en los mecanismos de seguridad
de Android y da lugar a un entendimiento más claro sobre el funcionamiento de la plataforma. Sin
embargo, no hay duda que el paso definitivo sería que el framework que construimos nos permita
establecer garantías sobre la implementación real de la plataforma. En pos de ese objetivo, hay
mucho trabajo por delante aún. Por ejemplo, es posible utilizar el programa extraído en
\textit{Haskell} para comparar los resultados producidos por las acciones reales de la plataforma
con las acciones del monitor de referencia. Esto permitiría monitorear las acciones que se llevan a
cabo en una implementación real de Android y evaluar, en tiempo real, si las políticas de seguridad
se cumplen o no. La metodología para lograr ese objetivo ya fue descripta en el trabajo a partir del
cual esta tesina comenzó\cite{luna-cleiej}.

En Septiembre del 2020, la versión 11 de Android fue lanzada oficialmente. Esta versión incluye
algunos cambios que siguen apostando a mejorar la seguridad del sistema de permisos, como por
ejemplo permisos que se eliminan automáticamente cuando las aplicaciones entrán en desuso por mucho
tiempo; o permisos que pueden utilizarse una única vez en los recursos más sensibles, como la cámara
o el micrófono.  Al momento de redactar esta tesina, la versión 12 de Android se encuentra en
\textit{beta}. En ella, se expande el comportamiento del restablecimiento automático de permisos que
se introdujo en Android 11 con el concepto ``el estado de hibernación''. Fundamentalmente, una
aplicación entra en estado de hibernación cuando ha estado en desuso por algunos meses. En ese
estado, la misma no podrá ejecutar servicios ni alertas en segundo plano. Estos cambios podrían ser
formalizados y agregados al modelo en el futuro.