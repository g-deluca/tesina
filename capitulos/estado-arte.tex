
\chapter{Estado del arte}
\label{chapter:estado}

En esta sección se presentan algunos trabajos académicos relacionados con el
modelo de seguridad de Android. Se organizará de la siguiente manera: primero,
se describen trabajos informales\footnote{Entendemos por trabajo informal a
aquellos que describen coloquialmente el sistema, por más rigurosa que sea la
descripción} que son de interés por abarcar características nuevas, por la
información que condensan o por ser pioneros en el área. Luego, se mencionan
varias herramientas (o aplicaciones) que utilizan técnicas de análisis estático
de código para detectar potenciales vulnerabilidades, todas desarrolladas dentro
de un marco académico. A pesar de que este enfoque  es distinto al que se
propone en esta tesina, se tomó la decisión de incluirlo en este capítulo ya que
conforma gran parte de la investigación actual y de los últimos años. Por
último, se analizan aquellos trabajos que utilizan métodos formales para dar una
especificación de la plataforma. Estos trabajos se describen con más detalle que
los anteriores e incluso se contemplan artículos que no son recientes pero
presentan un enfoque novedoso o relevante a este trabajo.

Dentro del primer grupo, se encuentran artículos como el de William Enck
\textit{et al.}~\cite{enck}: uno de los primeros trabajos académicos en
describir el modelo de seguridad de Android. Los autores buscaban
\textit{desenmascarar} la complejidad a la que debían enfrentarse los
desarrolladores cuando se proponían construir aplicaciones seguras. A pesar de
la antigüedad de este artículo, hoy en día sigue siendo relevante por la
explicación concreta y concisa sobre las bases del sistema de permisos de
Android. También se encuentra el trabajo de Wang y Wu~\cite{wang}, quienes
resumen las vulnerabilidades que existen en el sistema de comunicación entre
componentes, para luego discutir sobre el estado del arte en la detección y
prevención de las mismas.

Recientemente, René Mayrhofer \textit{et al.}~\cite{mayrhofer} realizaron un
trabajo similar para la versión 9.0 de Android. En él, se definen los principios
de seguridad del sistema en cuestión y un modelo de amenazas \textit{por capas},
a partir cual se derivan amenazas concretas que dependen de la cercanía entre
atacante y dispositivo móvil. Además, este artículo analiza los cambios que se
fueron introduciento en el sistema operativo para mitigar estas amenazas.

% Por otro lado, tenemos el trabajo de Yasemin Acar \textit{et al.}~\cite{sok},
% quienes presentan una visión sistémica e integradora de las distintas líneas
% de investigación en Android. Los autores realizan un análisis de los diversos
% enfoques desde los cuales se aborda la seguridad de esta plataforma,
% comparándolos, y estableciendo, según ellos, cuál es el camino para la
% investigación futura.

Este tipo de trabajos constituyen un complemento importante a la documentación
oficial de Android, brindándole referencias más claras a los desarrolladores y
nuevas herramientas que permiten resguardar sus aplicaciones. Un ejemplo de esto
fue el trabajo de Felt \textit{et al.}~\cite{felt}, quienes estudiaron un grupo
de aplicaciones disponibles para la versión 2.2 de Android y detectaron que
muchas de ellas pedían más permisos de los que realmente necesitaban. Los
autores investigaron las causas de sobreprivilegio de estas aplicaciones y
encontraron que, muchas veces, los desarrolladores intentaban otorgar la menor
cantidad de privilegios necesarios, pero en reiteradas ocasiones fallaban por
falta de una documentación precisa. En consecuencia, el grupo desarrolló
Stowaway, una de las primeras herramientas dedicadas a la detección de permisos
innecesarios.

Actualmente existe una gran cantidad de herramientas de análisis estático que
ayudan a detectar sobreprivilegios o un flujo de información indebido, siendo
las más recientes:
%
M-Perm~\cite{mperm},
%
IC3~\cite{ic3},
% que incorpora el concepto de \textit{propagación de constantes compuestas
% multi-valuadas} para lograr una mayor eficiencia;
Droidtector~\cite{droidtector},
% que a diferencia del resto no necesita el código fuente de las aplicaciones o
% de Android para realizar el análisis.
Covert~\cite{covert} y Separ~\cite{separ}.
% que combinan análisis estático con métodos formales para inferir
% automáticamente propiedades sobre un conjunto de aplicaciones, a partir de las
% cuales se derivarán políticas de seguridad
Las dos últimas combinan análisis estático con métodos formales para inferir
automáticamente propiedades sobre un conjunto de aplicaciones y luego derivar
políticas de seguridad. La gran oferta que existe de este tipo de aplicaciones,
da lugar a trabajos como el Lina Qiu \textit{et al.}~\cite{qiu}, en el que se
analizan y comparan las herramientas, que según los autores, son las más
destacadas: Flowdroid combinada con IccTA~\cite{iccta},
Amandroid~\cite{amandroid} y DroidSafe~\cite{droidsafe}.

Abordar la seguridad de Android con este enfoque presenta una diferencia
fundamental con respecto a la que se propone en esta tesina: estas herramientas
se focalizan en proteger una aplicación en particular (o en algunos casos, un
conjunto de aplicaciones); mientras que una especificación formal de la
plataforma subyacente permite extraer propiedades relevantes a \textbf{todas}
las aplicaciones y al sistema en general. Por otro lado, este tipo de
aplicaciones son un recurso más accesible para los usuarios que desean
resguardar información sensible sin ser especialistas en el área.

% A pesar de ser el enfoque con más publicaciones relacionadas a la seguridad de
% Android, presenta una diferencia fundamental con el que se propone en esta
% tesina: estas herramientas se focalizan en estudidar una aplicación (o en
% algunos casos, un conjunto de aplicaciones)


% Las herramientas e este tipo se focalizan en estudiar una aplicación en
% particular (o en algunos casos, un conjunto de aplicaciones) y extraer
% propiedades de seguridad que solo conciernen a ella. En cambio, en esta tesina
% estudiamos el sistema de permisos subyacente, extrayendo propiedades
% relevantes para todas las aplicaciones y para el sistema en general.


Entrando en el terreno de los métodos formales, nos encontramos en primer lugar
con el trabajo de Chaudhuri~\cite{chaudhuri}. En el mismo, se desarrolló un
lenguaje que permite describir un subconjunto de aplicaciones de Android y
razonar sobre ellas. Adicionalmente, se presentó un sistema de tipos para este
lenguaje y se demostró un teorema que garantiza que las aplicaciones bien
tipadas preservan la confidencialidad de los datos que manejan. Parcialmente
inspirado en este trabajo, Bugliesi \textit{et al.} desarrollaron
$\pi$-Perm~\cite{bugliesi}, un sistema de tipos y efectos que tiene como
finalidad detectar problemas de \textit{privilege escalation}. De manera análoga
al trabajo de Chaudhuri, una expresión bien tipada en $\pi$-Perm garantiza que
la aplicación real a la cual está representando, no es vulnerable al ataque
mencionado. Similarmente, Armando \textit{et al.} definen un
lenguaje~\cite{armando}, acompañado por su semántica operacional, que permite
describir interacciones entre aplicaciones. Al igual que en los trabajos
anteriores, se define un sistema de tipos y efectos; pero este está basado en un
formalismo del estilo del álgebra de procesos, conocido como \textit{history
expressions}~\cite{history-expressions}. Puesto en términos simples, una
\textit{history expression} sirve para representar los efectos laterales
vinculados a la seguridad del dispositivo, que se producen al realizar una
computación. Finalmente, los autores prueban que cualquier comportamiento que la
plataforma pueda tener en \textit{runtime}, está contenido en este modelo; y por
lo tanto, puede analizarse estáticamente.

Recientemente, Wilayat Khan \textit{et al.}~\cite{khan} retomaron el trabajo de
Chaudhuri y modelaron el lenguaje en él definido dentro del \textit{framework}
lógico-matemático Coq~\cite{coq}. De esta forma, pudieron no solo estudiar la
corrección y seguridad de las aplicaciones de manera mecánica y rigurosa, sino
que también utilizaron este asistente para probar la corrección -o
\textit{soundness}- del lenguaje en sí. En otro trabajo actual en el que
participó Khan~\cite{crashsafe}, se definió en Coq un modelo para estudiar el
sistema de comunicación entre componentes. El principal objetivo de este trabajo
es analizar la robustez de la plataforma cuando una aplicación detiene su
ejecución a causa de un fallo en la resolución de un \textit{intent}. A
diferencia del resto de los trabajos citados, éste se concentra en estudiar
propiedades de \textit{safety} y no de \textit{security}, a pesar de que este
sistema puede ser explotado para filtrar información sensible de los
usuarios~\cite{iccta}.

Por otra parte, Sadeghi \textit{et al.}~\cite{sadeghi-temp} presentan una
formalización de la plataforma escrita en TLA+, un lenguaje especificación
basado en la lógica lineal temporal~\cite{tla+}. Al incoporar el aspecto
temporal al modelo, los autores buscan definir propiedades cuya veracidad
dependa del momento en el que se la evalúe y de esta forma, modelar el
comportamiento del sistema a medida que evoluciona en el tiempo. Luego, proponen
un monitor de seguridad que otorga permisos temporales a las aplicaciones
siempre y cuando se cumplan todas las propiedades (o reglas) de seguridad
previamente definidas. Este permiso ``prestado'' es automáticamente revocado si
en algún momento el sistema se encuentra en un estado que compromete alguna de
las reglas.

Similarmente, Bagheri \textit{et al.}~\cite{bagheri15} proponen una
formalización del sistema de permisos de Android escrita en Alloy~\cite{alloy}:
un lenguaje basado en la lógica relacional de primer orden, que incorpora una
herramienta capaz de realizar análisis de satisfacibilidad automáticos sobre los
modelos en él descriptos. Con la ayuda de esta formalización, los autores
identificaron distintos tipos de vulnerabilidades que permiten esquivar el
chequeo de permisos. Particularmente, estudiaron la vulnerabilidad de permisos
personalizados, mediante la cual una aplicación maliciosa puede acceder a todos
los recursos de otra que estén protegidos por permisos personalizados. Esta
falla surge de que el sistema no impone restricciones con respecto al nombre de
los nuevos permisos que definen y, como consecuencia, dos permisos distintos
podrían tener el mismo nombre. Este trabajo luego se extendió para una nueva
versión de Android~\cite{bagheri}. La falla por permisos personalizados había
sido reportada previamente por Shin \textit{et al.}~\cite{shin-custom}. Una
diferencia fundamental entre este enfoque y el de esta tesina es el tipo de
análisis que se realizó. A pesar de que Alloy es capaz de producir
contraejemplos de manera automática, algo realmente útil a la hora de buscar
potenciales fallas; no es posible demostrar propiedades de una manera rigurosa y
formal.

En trabajos previos encabezados por Gustavo Betarte y Carlos
Luna~\cite{betarte-2017, betarte-2016, luna-cleiej}, se utilizó el asistente de
pruebas Coq para modelar un sistema de transición de estados que representa,
principalmente, los distintos estados que atraviesa la plataforma cuando se
realizan operaciones sobre ella (por ejemplo, al instalar o desinstalar una
aplicación). A partir de esta especificación, no solo se probaron propiedades
relevantes a la seguridad del modelo, sino que también se extrajo una
implementación certificada del mismo. Los autores explican cómo esta
implementación puede utilizarse para generar casos de pruebas abstractos dentro
del \textit{testing} basado en modelos, o bien, cómo puede usarse para
monitorear las acciones realizadas en un sistema real y evaluar si las
propiedades deseadas efectivamente se cumplen. Estos trabajos presentan el
modelo que será actualizado y extendido en esta tesina.

%%%%%%%%%%
% Fragmento tomado de \cite{luna-cleiej}
%%%
% In particular, Shin et al. [42, 43] build a formal framework that represents
% the Android permission system, which is based on the Calculus of Inductive
% Constructions and it is developed in Coq, as we do. However, that
% formalization does not consider, for instance, several aspects of the platform
% covered in our model, namely, the different types of components, the
% interaction between a running instance and the system, the reading/writing
% operation on a content provider and the semantics of the permission delegation
% mechanism. They also do not consider novel aspects of the Android security
% model, such as managing runtime permissions.
%%%
Un enfoque similar a este, es el de Wook Shin \textit{et al.}, quienes también
utilizaron Coq para modelar el sistema de permisos de Android~\cite{shin}. Sin
embargo, esta formalización no considera aspectos de la plataforma que sí son
considerados por los trabajos anteriores (y por ende, por esta tesina); como por
ejemplo, los distintos tipos de componentes, la interacción entre instancias de
aplicaciones en ejecución y el sistema, la operación de escritura/lectura en un
\textit{content provider} y la semántica del sistema de delegación de permisos.
Al mismo tiempo, cuando Android incorporó los permisos otorgados en
\textit{runtime}, este modelo no fue actualizado. El trabajo de Fragkaki \textit{et al.}
tambień presenta un modelo formal basado en transiciones de
estado~\cite{fragkaki}, pero el mismo no está desarrolado dentro de un
\textit{framework} que permita realizar pruebas asisistidas por computadora.
Además, el modelo se corresponde con una de las primeras versiones de Android,
por lo que tampoco contempla los cambios más recientes en el sistema de
permisos.