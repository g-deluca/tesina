
\chapter*{Estado del arte}
\label{chapter:estado}
\addcontentsline{toc}{chapter}{\nameref{chapter:estado}}

En esta sección se describen algunos trabajos sobre el modelo de seguridad de
Android. \textit{A priori}, podemos distinguir dos grandes grupos: aquellos que
reportan alguna falla puntual del sistema y los que buscan dar una descripción
más general del mismo, ya sea de manera formal o informal.

Dentro del primer grupo mencionado podemos encontrar bla bla bla. TODO: Buscar
trabajos que apunten a vulnerabilidades específicas. Antes me había concentrado
en lo que tiene que ver con modelos formales de Android. El paper de Bagheri
\cite{bagheri} tiene muchas referencias a este tipo de trabajos.

Por otra parte, el trabajo de Bagheri \textit{et. al} \cite{bagheri15} une a los
dos grupos distinguidos en el primer párrafo. Proponen, a diferencia de los trabajos
mencionados anteriormente, utilizar un modelo formal para encontrar
vulnerabilidades en el sistema de permisos de Android. El mismo está escrito en
Alloy \cite{alloy}, un lenguaje basado en lógica relacional de primer orden, con
un motor de análisis que realiza \textit{bounded verifiaction} de los
%
\todo{traducción de bounded verifiaction?}
%
modelos que en este lenguaje se describan.

Analizando este modelo, identificaron distintos tipos de vulnerabilidades que
permiten esquivar por completo el chequeo de permisos. Particularmente,
estudiaron la vulnerabilidad
%
\todo{mejor traducción para custom-permission?}
%
de \textit{permisos personalizados}, mediante la cual una aplicación maliciosa
puede acceder a todos los recursos de otra que estén protegidos por permisos
personalizados. Esta falla surge de que el sistema no impone restricciones con
respecto al nombre de los nuevos permisos que definen y, como consecuencia, dos
permisos distintos podrían tener el mismo nombre. En \cite{bagheri}, este trabajo
se extendió para una nueva versión de Android. La falla por permisos personalizados
ya había sido reportada por Shin \textit{et. al} en \cite{shin-custom}.

% Through an analysis of our model, we identified a number of vulnerabilities
% in the protocol that allow a malicious application to entirely bypass permission
% checks. In particular, we performed a study of a vulnerability that has not been
% studied in the security literature before—called the custom permission vulnerability. To confirm that an abstract attack scenario identified during the analysis
% is indeed realistic, we demonstrated the attack on concrete Android applications
% across different versions of Android. Through our study, we show that the custom permission vulnerability is widespread, and that many popular applications
% are, in fact, susceptible to this type of attacks.

