
\chapter*{Estado del arte}
\label{chapter:estado}
\addcontentsline{toc}{chapter}{\nameref{chapter:estado}}

En esta sección se describen algunos trabajos sobre el modelo de seguridad de
Android. \textit{A priori}, podemos distinguir dos grandes grupos: aquellos que
reportan alguna falla puntual del sistema y los que buscan dar una descripción
más general del mismo, ya sea de manera formal o informal\footnote{Entendemos
por enfoque informal a aquellos enfoques que no utilizan métodos formales para
el estudio de la plataforma.}.

\subsection*{Fallas puntuales}

Dentro del primer grupo mencionado podemos encontrar bla bla bla. TODO: Buscar
trabajos que apunten a vulnerabilidades específicas. El paper de Bagheri
\cite{bagheri} tiene muchas referencias a este tipo de trabajos.

\subsection*{Enfoque informal}

Por otro lado, encontramos trabajos como el de William Enck
\textit{et al.}~\cite{enck}, uno de los primeros artículos académicos en
describir el modelo de seguridad de Android. A través de esta descripción, los
autores buscaban \textit{desenmascarar} la complejidad a la que debían
enfrentarse los desarrolladores a la hora de construir aplicaciones seguras.

\todo{Leer con más detalle \cite{sok} y ver si vale la pena profundizar}
Dentro de esta línea también se encuentra el trabajo de Yasemin Acar \textit{et
al.}~\cite{sok}, quienes presentan una visión sistémica e integradora de las
distintas líneas de investigación en seguridad de Android. Los autores realizan
un análisis de los distintos enfoques desde los cuales se aborda la seguridad de
este tipo de sistemas, comparándolos y estableciendo, según ellos, las bases para
que la investigación futura pueda ser unificada.

Recientemente, René Mayrhofer \textit{et al.}~\cite{mayrhofer} publicaron un
trabajo de características similares a los anteriores basado en la versión 9.0
de Android. El mismo, además de dar un descripción detallada del modelo, incluye
una discusión de sus implicaciones y un posterior análisis sobre las medidas que
se tomaron a lo largo del tiempo para mitigar distintas amenazas.

Este tipo de trabajos constituyen un complemento importante a la documentación
oficial de Android, brindándole nuevas herramientas y referencias más claras a
los desarrolladores. Un ejemplo de esto fue el trabajo de Felt \textit{et
al.}~\cite{felt}, quienes estudiaron un grupo de aplicaciones disponibles para
la versión 2.2 de Android y detectaron que muchas de ellas pedían más permisos
de los que realmente necesitaban. Los autores investigaron las causas de
sobreprivilegio de estas aplicaciones y encontraron que muchas veces, los
desarrolladores intentaban otorgar la menor cantidad de privilegios necesarios
pero en reiteradas ocasiones fallaban por falta de una documentación precisa. En
consecuencia, el grupo desarrolló Stowaway, una de las primeras herramientas
dedicadas a la detección de permisos innecesarios.

Actualmente existe una gran cantidad de herramientas de análisis estático,
siendo las más recientes: M-Perm~\cite{mperm}; IC3~\cite{ic3}, que incorpora el
concepto de \textit{propagación de constantes compuestas multi-valuadas} para
lograr una mayor eficiencia; Covert~\cite{covert} y Separ~\cite{separ}, que
combinan análisis estático con métodos formales para inferir automáticamente
propiedades sobre un conjunto de aplicaciones, a partir de las cuales se
derivarán políticas de seguridad; y Droidtector~\cite{droidtector}, que a
diferencia del resto no necesita el código fuente de las aplicaciones o de
Android. Las herramientas de este tipo se focalizan en estudiar una aplicación en
particular (o en algunos casos, un conjunto de aplicaicones) y extraer
propiedades de seguridad que solo conciernen a ella. En cambio, en esta tesina
estudiamos el sistema de permisos subyacente, extrayendo propiedades relevantes
para todas las aplicaciones y para el sistema en general.


\subsection*{Enfoque formal}

Entrando en el terreno de los métodos formales, Chaudhuri~\cite{chaudhuri}
presentó un lenguaje tipado que permite describir un subconjunto de aplicaciones
de Android y razonar sobre ellas. Adicionalmente, presentó un sistema de tipos
para este lenguaje, garantizando que las aplicaciones bien tipadas preservan la
confidencialidad de los datos que manejan. Recientemente, Wilayar Khan
\textit{et al.}~\cite{khan} formalizaron y demostraron la corrección de este
sistema de tipos utilizando el asistente de pruebas Coq.

En otro trabajo reciente encabezado por W. Khan~\cite{crashsafe}, se definió en
Coq un modelo para estudiar el sistema de comunicación entre componentes. El
principal objetivo de este trabajo es analizar la robustez de la plataforma
cuando una aplicación detiene su ejecución a causa de un fallo en la resolución
de un \textit{intent}. A diferencia del resto de los trabajos citados, éste se
concentra en estudiar propiedades de \textit{safety} y no de \textit{security}, a
pesar de que este sistema puede ser explotado para filtrar información sensible de
los usuarios~\cite{iccta}.

Por otra parte, Bagheri \textit{et al.}~\cite{bagheri15} proponen una
formalización del sistema de permisos de Android escrita en Alloy~\cite{alloy},
un lenguaje basado en la lógica relacional de primer orden,
%
\todo{traducción de bounded verifiaction?}
%
análisis capaz de realizar \textit{bounded verification} de los modelos que en
este lenguaje se describan. Con la ayuda de este modelo, los autores
identificaron distintos tipos de vulnerabilidades que permiten esquivar por
completo el chequeo de permisos. Particularmente, estudiaron la vulnerabilidad
de permisos personalizados, mediante la cual una aplicación maliciosa puede
acceder a todos los recursos de otra que estén protegidos por permisos
personalizados. Esta falla surge de que el sistema no impone restricciones con
respecto al nombre de los nuevos permisos que definen y, como consecuencia, dos
permisos distintos podrían tener el mismo nombre. Este trabajo luego se extendió
para una nueva versión de Android~\cite{bagheri}. La falla por permisos
personalizados ya había sido reportada por Shin \textit{et al.}
en~\cite{shin-custom}.

\todoCarlos{debería profundizar más en este párrafo?}
Una diferencia fundamental entre este enfoque y el de esta tesina es el tipo de
análisis que se realizó. A pesar de que Alloy es capaz de producir
contraejemplos de manera automática, algo realmente útil a la hora de buscar
potenciales fallas; no es posible demostrar propiedades de una manera rigurosa y
formal.

En trabajos previos encabezados por Carlos Luna~\cite{betarte, luna-cleiej}, se
utilizó el asistente de pruebas Coq~\cite{coq} para dar una descripción formal y
exhaustiva del sistema de permisos de Android. A partir de esta especificación,
no solo se probaron propiedades relevantes a la seguridad del modelo, sino que
también se extrajo una implementación certificada del mismo. Los autores explican
cómo esta implementación puede utilizarse para generar casos de pruebas abstractos
dentro del \textit{testing} basado en modelos, o bien, cómo puede usarse para
monitorear las acciones realizadas en un sistema real y evaluar si las propiedades
deseadas efectivamente se cumplen. Estos trabajos presentan el modelo que será
actualizado y expandido en esta tesina.