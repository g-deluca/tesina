
\chapter*{Estado del arte}
\label{chapter:estado}
\addcontentsline{toc}{chapter}{\nameref{chapter:estado}}

En esta sección se describen algunos trabajos sobre el modelo de seguridad de
Android. \textit{A priori}, podemos distinguir dos grandes grupos: aquellos que
reportan alguna falla puntual del sistema y los que buscan dar una descripción
más general del mismo, ya sea de manera formal o informal\footnote{Entendemos
por enfoque informal a aquellos enfoques que no utilizan métodos formales para
el estudio de la plataforma.}.

\subsection*{Fallas puntuales}

Dentro del primer grupo mencionado podemos encontrar bla bla bla. TODO: Buscar
trabajos que apunten a vulnerabilidades específicas. El paper de Bagheri
\cite{bagheri} tiene muchas referencias a este tipo de trabajos.

\subsection*{Enfoque formal}

Por otra parte, el trabajo de Bagheri \textit{et al.} \cite{bagheri15} une a los
dos grupos distinguidos en el primer párrafo. Proponen, a diferencia de los trabajos
mencionados anteriormente, utilizar un modelo formal para encontrar
vulnerabilidades en el sistema de permisos de Android. El mismo está escrito en
Alloy \cite{alloy}, un lenguaje basado en lógica relacional de primer orden, con
un motor de análisis que realiza \textit{bounded verifiaction} de los
%
\todo{traducción de bounded verifiaction?}
%
modelos que en este lenguaje se describan.

Analizando este modelo, identificaron distintos tipos de vulnerabilidades que
permiten esquivar por completo el chequeo de permisos. Particularmente,
estudiaron la vulnerabilidad
%
\todo{mejor traducción para custom-permission?}
%
de \textit{permisos personalizados}, mediante la cual una aplicación maliciosa
puede acceder a todos los recursos de otra que estén protegidos por permisos
personalizados. Esta falla surge de que el sistema no impone restricciones con
respecto al nombre de los nuevos permisos que definen y, como consecuencia, dos
permisos distintos podrían tener el mismo nombre. En~\cite{bagheri}, este trabajo
se extendió para una nueva versión de Android. La falla por permisos personalizados
ya había sido reportada por Shin \textit{et al.} en~\cite{shin-custom}.

\subsection*{Enfoque informal}

Alejándonos del terreno de los métodos formales, encontramos trabajos como el de
William Enck \textit{et.al}~\cite{enck}, uno de los primeros artículos académicos
en describir el modelo de seguridad de Android. A través de esta descripción, los
autores buscaban \textit{desenmascarar} la complejidad a la que debían enfrentarse
%
\todo{traducción literal de unmask}
%
los desarrolladores a la hora de construir aplicaciones seguras.

Recientemente, René Mayrhofer \textit{et al.}~\cite{mayrhofer} publicaron un
%
\todo{llenar gap temporal con algún trabajo más, puede ser \cite{sok}}
%
trabajo de similar basado en la versión 9.0 de Android. El mismo, además de dar
un descripción detallada del modelo, incluye una discusión de sus implicaciones
y un posterior análisis sobre las medidas que se tomaron a lo largo del tiempo
para mitigar distintas amenazas.

Este tipo de trabajos constituyen un complemento importante a la documentación
oficial de Android,
%
\todo{citarla?}
%
brindándole nuevas herramientas y referencias más claras a los desarrolladores.
Un ejemplo de esto fue el trabajo de Felt \textit{et al.}~\cite{felt}, quienes
estudiaron un grupo de aplicaciones disponibles para la versión 2.2 de Android y
detectaron que muchas de ellas pedían más permisos de los que realmente
necesitaban. Los autores investigaron las causas de sobreprivilegio de estas
aplicaciones y encontraron que muchas veces, los desarrolladores intentaban
otorgar la menor cantidad de privilegios necesarios pero en reiteradas ocasiones
fallaban por falta de una documentación precisa. En consecuencia, el grupo
desarrolló Stowaway, una herramienta pionera en la detección de permisos
innecesarios. En esta línea encontramos también el trabajo de Kathy Wain Yee Au
\textit{et al.}~\cite{pscout}, el del Piper Chester \textit{et
al.}~\cite{mperm}, el de Alexandre Bartel \textit{et al.}~\cite{bartel}, y el de
Sha Wu y Jiajia Liu~\cite{droidtector}; siendo este último el más reciente,
realizado con la intención de contrarrestar las limitaciones detectadas en los
trabajos previos.
%
\todo{en \cite{droidtector} se comparan todos estos trabajos, incorporo esa comparación acá?}
%
