\documentclass[a4paper]{article}

\usepackage[spanish]{babel}
\usepackage[utf8]{inputenc}

\usepackage[backend=biber, maxbibnames=4]{biblatex}
% \usepackage{amsmath}
% \usepackage{amsfonts}
% \usepackage{amssymb}
% \usepackage{amsthm}
% \usepackage{enumerate}
% \usepackage{color}
% \usepackage{graphicx}
\usepackage{todonotes}
\usepackage[hidelinks]{hyperref}

\newcommand{\todoGuido}[1]{\todo[color=red]{Guido: {#1}}}

\addbibresource{referencias.bib}

\begin{document}

\title{Propuesta de Tesina para la obtención del grado Licenciado en Ciencias de la Computación}
\maketitle

\begin{description}
    \item[Postulante: ] Guido De Luca
    \item[Director: ] Carlos Luna
    \item[Codirector: ] Dante Zanarini
\end{description}

\section{Situación del postulante}
Al momento de presentar la propuesta, el estudiante se encuentra realizando el
trabajo final de la materia Compiladores. El resto de las materias se encuentran
rendidas y aprobadas. Se espera dedicar al menos 20 horas semanales a la realización
de la tesina.

\section{Título}
\todo{¿?}
Extensión de la especificación de un prototipo certificado del sistema de permisos
de Android

\section{Motivación y Objetivo General}

Android es uno de los sistemas operativos para dispositivos móviles más
utilizados en el mundo \cite{1,2}. Existen actualmente cerca de 2,7 millones de
aplicaciones en su tienda oficial, destinadas a cubrir una gran cantidad de
aspectos de la vida cotidiana moderna \cite{3}. Muchas de ellas pueden ser críticas
en lo que a privacidad y seguridad respecta, y por lo tanto, esperan que la plataforma
les brinde los recursos necesarios para poder resguardar la información sensible
que manejan.
% Muchas de ellas pueden ser
% críticas en lo que a seguridad y privacidad respecta, y Android, como sistema
% operativo, es el encargado de otorgarle a los usuarios y a los desarrolladores
% las garantías necesarias para que la privacidad de los.
En pos de lograr este objetivo, Android se basa fundamentalmente en un
sistema de consentimiento \textit{multi-parte}, donde una acción ocurre solo si
todas las partes involucradas están de acuerdo. Por ejemplo, si una aplicación
quisiera compartir una imagen con otra, se necesitaría el consentimiento de los
siguientes actores:

\begin{itemize}
   \item \textbf{El usuario: } Debe seleccionar la aplicación a la cual se desea
   compartir la imagen, utilizando la interfaz gráfica.
   \item \textbf{Los desarrolladores: } El desarrollador de la aplicación
   elegida por el usuario debe aceptar los datos que está por recibir. Al mismo
   tiempo, el desarrollador de la aplicación que compartirá la imagen debió
   haberle otorgado los permisos necesarios para hacerlo.
   \item \textbf{La plataforma: } Es la encargada de arbitrar el acceso a la
   información y de garantizar que la aplicación que recibirá la imagen solo pueda
   acceder a la información que se está compartiendo explícitamente.
\end{itemize}


Estudiar formalmente estas políticas de seguridad resulta fundamental para
lograr un entendimiento preciso sobre qué es lo que se quiere o espera del
sistema. Particularmente, una especificación formal permite demostrar
propiedades críticas del sistema o en algunos casos encontrar vulnerabilidades
existentes \cite{alloy}.

El objetivo de este trabajo es realizar una especificación, partiendo de una
preexistente, lo más abarcativa posible, formal y robusta que permita verificar
el modelo de seguridad de Android en su última versión.


\section{Fundamentos y estado del conocimiento sobre el tema}

Android es un sistema operativo de \textit{código abierto} diseñado
originalmente para dispositivos móviles y desarrollado por Google junto a la
Open Handset Alliance (OHA). Actualmente, el uso de Android como sistema
operativo puede encontrarse también en dispositivos como tablets,
\textit{smartwatches}, televisores \textit{inteligentes} o incluso en autos
\cite{android-car}.

Una característica fundamental de esta plataforma es que cualquier aplicación,
sea principal\footnote{Entendemos como aplicación principal a aquellas
pre-instaladas con el sistema operativo} o creada por algún desarrollador,
puede, con los permisos adecuados, utilizar tanto los recursos/servicios del
dispositivo móvil como los ofrecidos por el resto de las aplicaciones.

Esta última característica sumada a la popularidad alcanzada por el sistema lo
convierte en un objeto de estudio interesante, pues una falla de seguridad en la
plataforma podría afectar a una gran cantidad de usuarios y desarrolladores.
Particularmente, estudiar el mecanismo de delegación de permisos es importante
para garantizarle a los desarrolladores una documentación precisa sobre el mismo
y permitirles una construcción segura de sus aplicaciones, al mismo tiempo que
permite encontrar vulnerabilidades existentes o prevenir futuras.

El estudio de este modelo se hará desde una perspectiva formal (utilizando una
especificación desarrollada en Coq) por la ventajas que confiere: poder definir
qué es lo que se espera del sistema sin ambigüedades y tener, no solo la
capacidad de probar que ciertas propiedades de seguridad son correctas, sino
también una prueba tangible y computable de las mismas. Contar con un modelo
formalmente especificado de los mecanismos de seguridad de esta plataforma
permite formular precisamente las propiedades que la misma debe garantizar,
validar su satisfacción, razonar sobre el comportamiento esperado de la misma y
eventualmente derivar o inferir la necesidad de mecanismos y/o propiedades
previamente no identificados.

El principal objetivo de esta tesina es realizar un análisis exhaustivo de las
nuevas características del modelo de seguridad de Android, enfocándose
principalmente en el sistema de permisos. Se partirá de una primera
especificación formal realizada a partir de otros dos proyectos de
grado~\cite{fgorostiaga, aromano} dirigidos por un miembro del actual equipo,
utilizando el asistente de pruebas Coq~\cite{coq}. Esta especificación se
focaliza en el mecanismo de delegación de permisos y en la interacción con el
framework de aplicaciones para realizar llamadas al sistema.

En este trabajo proponemos una extensión conservativa de dicho modelo. En primer
lugar, veremos qué elementos del modelo de seguridad formalizado deberían
modificarse o extenderse para contemplar las versiones más recientes de Android.
Luego, revisaremos que en este nuevo modelo las propiedades que ya han sido
demostradas se adapten correctamente. Por último, realizaremos un análisis de
las nuevas propiedades inherentes a los cambios introducidos.

La extensión se realizará utilizando el mismo framework lógico-matemático
provisto por el Cálculo de Construcciones Inductivas \cite{coqart, coquand:88},
en el asistente Coq, por lo que será posible derivar una versión actualizada del
oráculo para el análisis de implementaciones Android existentes obtenido en
\cite{fgorostiaga}.

\section{Objetivos específicos}
\begin{enumerate}
    \item Extender la especificación desarrollada en un proyecto de grado
          anterior, incorporando:
          \label{item:especificacion-coq}
          \begin{itemize}
              \item Caracterísitcas de seguridad de las últimas versiones de
                    Android (desde Android 6.0 hasta Android 9.0).
              \item Revisión de propiedades ya demostradas en el modelo con las
                    nuevas caracterísitcas.
              \item Análisis de nuevas propiedades de seguridad relacionadas a
                    las nuevas caracterísitcas del modelo.
          \end{itemize}
    \item Extender el prototipo funcional certificado del modelo desarrollado
          con las nuevas caracterísitcas.
          \label{item:prototipo}
\end{enumerate}

\section{Metodología y plan de trabajo}

Para alcanzar el objetivo específico \ref{item:especificacion-coq} se pondrá a
disposición del estudiante el modelo formal ya realizado para Android
Marshmallow. Al mismo tiempo, se discutirán qué extensiones realizar y qué
propiedades considerar mediante comunicaciones periódicas con los tutores de
trabajo.

Para el \hyperref[item:prototipo]{segundo} objetivo específico se brindará una
descripción concreta del resultado esperado.

\subsection{Programa tentativo de trabajo}
\begin{itemize}
    \item Redacción de un estado del arte:
          \todo{Suena raro porque ya está escrito acá, no?}
    \item Analizar la especificación ya existente: ? semanas
    \item Ver aspectos a modificar y extensiones a realizar en el estado y las
          acciones del modelo: ? semanas
    \item Revisar las propiedades ya consideradas: ? semanas
    \item Plantear y demostrar nuevas propiedades relacionadas a las nuevas
          características del modelo: ? semanas
    \item Obtener un prototipo certificado del modelo: ? semanas
\end{itemize}

\printbibliography


\end{document}