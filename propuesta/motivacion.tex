\section{Motivación y Objetivo General}

Android es uno de los sistemas operativos para dispositivos móviles más
utilizados en el mundo \cite{1,2}. Existen actualmente cerca de 2,7 millones de
aplicaciones en su tienda oficial, destinadas a cubrir una gran cantidad de
aspectos de la vida cotidiana moderna \cite{3}. Muchas de ellas pueden ser críticas
en lo que a privacidad y seguridad respecta, y por lo tanto, esperan que la plataforma
les brinde los recursos necesarios para poder resguardar la información sensible.
% Muchas de ellas pueden ser
% críticas en lo que a seguridad y privacidad respecta, y Android, como sistema
% operativo, es el encargado de otorgarle a los usuarios y a los desarrolladores
% las garantías necesarias para que la privacidad de los.
En pos de lograr este objetivo, Android se basa fundamentalmente en un
sistema de consentimiento \textit{multi-parte}, donde una acción ocurre solo si
todas las partes involucradas están de acuerdo. Por ejemplo, si una aplicación
quisiera compartir una imagen con otra, se necesitaría el consentimiento de los
siguientes actores:

\begin{itemize}
   \item \textbf{El usuario: } Debe seleccionar la aplicación a la cual se desea
   compartir la imagen, utilizando la interfaz gráfica.
   \item \textbf{Los desarrolladores: } El desarrollador de la aplicación
   elegida por el usuario debe aceptar los datos que está por recibir. Al mismo
   tiempo, el desarrollador de la aplicación que compartirá la imagen debió
   haberle otorgado los permisos necesarios para hacerlo.
   \item \textbf{La plataforma: } Es la encargada de arbitrar el acceso a la
   información y de garantizar que la aplicación que recibirá la imagen solo pueda
   acceder a la información que se está compartiendo explícitamente.
\end{itemize}


Estudiar formalmente estas políticas de seguridad resulta fundamental para
lograr un entendimiento preciso sobre qué es lo que se quiere o espera del
sistema. Particularmente, una especificación formal permite demostrar
propiedades críticas del sistema o en algunos casos encontrar vulnerabilidades
existentes \cite{alloy}.

El objetivo de este trabajo es realizar una especificación, partiendo de una
preexistente, lo más abarcativa posible, formal y robusta que permita verificar
el modelo de seguridad de Android en su última versión.
