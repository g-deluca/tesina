\section{Fundamentos y estado del conocimiento sobre el tema}

Android es un sistema operativo de \textit{código abierto} diseñado
originalmente para dispositivos móviles y desarrollado por Google junto a la
Open Handset Alliance (OHA). Actualmente, el uso de Android como sistema
operativo puede encontrarse también en dispositivos como tablets,
\textit{smartwatches}, televisores \textit{inteligentes} o incluso en autos
\cite{android-car}.

Una característica fundamental de esta plataforma es que cualquier aplicación,
sea principal\footnote{Entendemos como aplicación principal a aquellas
pre-instaladas con el sistema operativo} o creada por algún desarrollador,
puede, con los permisos adecuados, utilizar tanto los recursos/servicios del
dispositivo móvil como los ofrecidos por el resto de las aplicaciones.

Esta última característica sumada a la popularidad alcanzada por el sistema lo
convierte en un objeto de estudio interesante, pues una falla de seguridad en la
plataforma podría afectar a una gran cantidad de usuarios y desarrolladores.
Particularmente, estudiar el mecanismo de delegación de permisos es importante
para garantizarle a los desarrolladores una documentación precisa sobre el mismo
y permitirles una construcción segura de sus aplicaciones, al mismo tiempo que
permite encontrar vulnerabilidades existentes o prevenir futuras.

El estudio de este modelo se hará desde una perspectiva formal (utilizando una
especificación desarrollada en Coq) por la ventajas que confiere: poder definir
qué es lo que se espera del sistema sin ambigüedades y tener, no solo la
capacidad de probar que ciertas propiedades de seguridad son correctas, sino
también una prueba tangible y computable de las mismas. Contar con un modelo
formalmente especificado de los mecanismos de seguridad de esta plataforma
permite formular precisamente las propiedades que la misma debe garantizar,
validar su satisfacción, razonar sobre el comportamiento esperado de la misma y
eventualmente derivar o inferir la necesidad de mecanismos y/o propiedades
previamente no identificados.

El principal objetivo de esta tesina es realizar un análisis exhaustivo de las
nuevas características del modelo de seguridad de Android, enfocándose
principalmente en el sistema de permisos. Se partirá de una primera
especificación formal realizada a partir de otros dos proyectos de
grado~\cite{fgorostiaga, aromano} dirigidos por un miembro del actual equipo,
utilizando el asistente de pruebas Coq~\cite{coq}. Esta especificación se
focaliza en el mecanismo de delegación de permisos y en la interacción con el
framework de aplicaciones para realizar llamadas al sistema.

En este trabajo proponemos una extensión conservativa de dicho modelo. En primer
lugar, veremos qué elementos del modelo de seguridad formalizado deberían
modificarse o extenderse para contemplar las versiones más recientes de Android.
Luego, revisaremos que en este nuevo modelo las propiedades que ya han sido
demostradas se adapten correctamente. Por último, realizaremos un análisis de
las nuevas propiedades inherentes a los cambios introducidos.

La extensión se realizará utilizando el mismo framework lógico-matemático
provisto por el Cálculo de Construcciones Inductivas \cite{coqart, coquand:88},
en el asistente Coq, por lo que será posible derivar una versión actualizada del
oráculo para el análisis de implementaciones Android existentes obtenido en
\cite{fgorostiaga}.