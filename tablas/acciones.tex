\begin{table}
    \label{table:actions}
    \begin{tabularx}{\linewidth}{|l X|}
        \hline
        \textbf{Acción}                    & \textbf{Descripción}                                                                                                                                                                                                 \\
        \hline
        $\mathtt{install}~app~m~c~res$     & Instala la aplicación con identificador $app$, cuyo manifiesto es $m$, su certificado es $c$ y la lista de recursos es $res$.                                                                                        \\
        \hline
        $\mathtt{uninstall}~app$           & Desinstala la aplicación con identificador $app$.                                                                                                                                                                    \\
        \hline
        $\mathtt{read}~ic~cp~u$            & El componente en ejecución $ic$ lee el recurso correspondiente al identificador URI $u$ del proveedor de contenido $cp$.                                                                                             \\
        \hline
        $\mathtt{write}~ic~cp~u~val$       & El componente en ejecución $ic$ escribe el valor $val$ en el recurso correspondiente al identificador $u$ del proveedor de contenido $cp$.                                                                           \\
        \hline
        $\mathtt{startActivity}~i~ic$      & El componente en ejecución $ic$ solicita comenzar la actividad especificada por el intent $i$.                                                                                                                       \\
        \hline
        $\mathtt{startActivityRes}~i~n~ic$ & El componente en ejecución $ic$ solicita comenzar la actividad especificada por el intent $i$ y espera como respuesta un token $n$.                                                                                  \\
        \hline
        $\mathtt{startService}~i~ic$       & El componente en ejecución $ic$ solicita comenzar el servicio especificado por el intent $i$.                                                                                                                        \\
        \hline
        $\mathtt{sendBroadcast}~i~ic~p$    & El componente en ejecución $ic$ envía el intent $i$ en modo \textit{broadcast}, especificando que solo los componentes que posean el permiso $p$ pueden recibirlo.                                                   \\
        \hline
        $\mathtt{sendOrdBroadcast}~i~ic~p$ & El componente en ejecución $ic$ envía el intent $i$ en modo \textit{broadcast} ordenado,  especificando que solo los componentes que posean el permiso $p$ pueden recibirlo.                                         \\
        \hline
        $\mathtt{sendSBroadcast}~i~ic$     & El componente en ejecución $ic$ envía el intent $i$ en modo \textit{sticky broadcast}.                                                                                                                               \\
        \hline
        $\mathtt{resolveIntent}~i~app$     & La aplicación $app$ vuelve al intent $i$ explícito.                                                                                                                                                                  \\
        \hline
        $\mathtt{stop}~ic$                 & El componente en ejecución $ic$ termina su ejecución.                                                                                                                                                                \\
        \hline
        $\mathtt{grantP}~ic~cp~app~u~op$   & El componente en ejecución $ic$ delega permisos permantentes a la aplicación $app$. Esta delegación autoriza a $app$ a realizar la operación $op$ en el recurso asignado al URI $u$ del proveedor de contenido $cp$. \\
        \hline
        $\mathtt{revokeDel}~ic~cp~u~op$    & El componente en ejecución $ic$ revoca los permisos otorgados al recurso $u$ del proveedor de contenidos $cp$ para realizar la operación $op$.                                                                       \\
        \hline
        $\mathtt{call}~ic~sac$             & El componente en ejecución $ic$ realiza el llamado a una función del sistema denominada $sac$.                                                                                                                       \\
        \hline
        $\mathtt{grant}~p~app$             & Otorga el permiso $p$ a la aplicación $app$ con la confirmación del usuario.                                                                                                                                         \\
        \hline
        $\mathtt{grantAuto}~p~app$         & Otorga automáticamente el permiso $p$ a la aplicación $app$ (sin requerir confirmación del usuario).                                                                                                                 \\
        \hline
        $\mathtt{revoke}~p~app$            & Revoca un permiso no agrupado $p$ de la aplicación $app$.                                                                                                                                                            \\
        \hline
        $\mathtt{revokePermGroup}~g~app$   & Revoca todos los permisos pertenecientes al grupo $g$ de la aplicación $app$.                                                                                                                                        \\
        \hline
        $\mathtt{hasPermission}~p~app$     & Chequea si la aplicación $app$ posee el permiso $p$.                                                                                                                                                                 \\
        \hline
        $\mathtt{receiveIntent}~i~ic~app$  & La aplicación $app$ recibe el intent $i$, enviado por el componente en ejecución $ic$.                                                                                                                               \\
        \hline
        $\mathtt{verifyOldApp}~app$        & El usuario verifica los permisos que han sido otorgados a la aplicación $app$. Solo se utiliza para aquellas aplicaciones que fueron instaladas previamente a la versión 6 de Android.                               \\
        \hline
    \end{tabularx}
    \caption{Acciones del sistema}
\end{table}